\documentclass[a4paper]{article}
\usepackage{color}              %Farben, f.r \definecolor{}
\usepackage{amssymb}            %Mathematische Symbole
\usepackage{amsthm}             %Besseres \newtheorem
\usepackage{amsmath}           %Mathematische Umgebungen
\usepackage{mathtools}          %\xRightarrow, etc
\usepackage{mathrsfs}           %enthaelt \mathscr
\usepackage{graphicx}
\usepackage{enumerate}          % in-place numerations def.
\usepackage{enumitem}
\usepackage{fullpage}
\usepackage{glossaries}
\usepackage[utf8]{inputenc}
\usepackage{array}
%\usepackage{multicol}
%\usepackage[notref,notcite]{showkeys}
%\usepackage{algorithm,algorithmic}
\usepackage{color}

\usepackage{graphicx}
\usepackage{xypic}
\entrymodifiers={+!!<0pt,\fontdimen22\textfont2>}
\usepackage[all]{xy}

\usepackage{float}
\usepackage{tikz}
\usepackage{tikz-cd}
\usepackage{tikz,fullpage}
\usetikzlibrary{arrows,%
                petri,%
                topaths}%
\usepackage{tkz-berge}
\usepackage[position=top]{subfig}
\usetikzlibrary{shapes.geometric}
\usetikzlibrary{decorations.markings}
\usetikzlibrary{matrix}
\usetikzlibrary{positioning,shapes,fit,arrows}
\newtheoremstyle{myremark} % name
    {7pt}                    % Space above
    {7pt}                    % Space below
    {}  	                 % Body font
    {}                           % Indent amount
    {\bf}       	         % Theorem head font
    {.}                          % Punctuation after theorem head
    {.5em}                       % Space after theorem head
    {}  % Theorem head spec (can be left empty, meaning ‘normal’)

\theoremstyle{plain}
\newtheorem{lemma}{Lemma}
\newtheorem{theorem}[lemma]{Theorem}
\newtheorem{fact}[lemma]{Fact}
\newtheorem{definition}[lemma]{Definition}
\newtheorem{corollary}[lemma]{Corollary}
\newtheorem{proposition}[lemma]{Proposition}
\newtheorem{conjecture}[lemma]{Conjecture}
\newtheorem{observation}[lemma]{Observation}
\newtheorem{problem}[lemma]{Problem}
\newtheorem{notation}[lemma]{Notation}
\newtheorem*{claim}{Claim}

\theoremstyle{myremark}
\newtheorem{remark}[lemma]{Remark}
\newtheorem{example}[lemma]{Example}
\newtheorem{exercise}[lemma]{Exercise}
\newtheorem{algorithm}[lemma]{Algorithm}
\newtheorem{application}[lemma]{Application}
\newtheorem*{goal}{Goal}

%%%%%% EDIT HERE: %%%%%%%%%%%
\newcommand{\LECTURENUMBER}{0}
\newcommand{\LECTURETITLE}{Short title}
\newcommand{\LECTURESCRIBE}{Your name}

%% Dokument Beginn %%%%%%%%%%%%%%%%%%%%%%%%%%%%%%%%%%%%%%%%%%%%%%%%%%%%%%%%
\begin{document}
\thispagestyle{empty}

\begin{center}
	{\Large\bf Graph coloring}\\
	{\bf Lecture notes, vol.9 \\ Edge Coloring}\\
\end{center}
Lecturer: Michal Adamaszek \hfill Scribe: Sokratis Theodoridis
\begin{center}
\line(1,0){450}
\end{center}

%%%%%%% EDIT ALSO BELOW: %%%%%%%%%%%%%%%%

In the next pages, $G$ is always a graph, $V(G)$ its set of vertices and $E(G)$ its set of edges.
\begin{definition}
An \emph{edge coloring} of $G$ is a function $f:E(G)\rightarrow C$ (for some set of colors $C$) such that if $e_1e_2$ have a common endpoint then $f(e_1)\neq f(e_2)$. The \emph{edge chromatic number} (\emph{chromatic index}) $\chi'(G)$ is the smallest $k$ such that $G$ has an edge-coloring with colors $\{1,\dots,k\}$.
\end{definition}
\begin{center}
\begin{tikzpicture} 
    \draw  
        node[fill=black,circle,inner sep=0pt,minimum size=3pt] (1) at (0,0) {}
        node[fill=black,circle,inner sep=0pt,minimum size=3pt] (2) at (0,-2) {}
        node[fill=black,circle,inner sep=0pt,minimum size=3pt] (3) at (2,0) {}
        node[fill=black,circle,inner sep=0pt,minimum size=3pt] (4) at (-2,-1) {}
        node[fill=black,circle,inner sep=0pt,minimum size=3pt] (5) at (2,-2) {}
  ; 
    \draw [color=blue] (1) -- (2) node {} (0,-1) node [text=blue,right] {$a$};
    \draw [color=red] (1) -- (3) node {} (1,0) node [text=red,above] {$b$};
    \draw [color=green] (1) -- (4) node {} (-1,-0.9) node [text=green,above] {$c$};
    \draw [color=blue] (3) -- (5) node {} (2,-1) node [text=blue,right] {$a$};
    \draw [color=green] (5) -- (2) node {} (1,-2) node [text=green,below] {$c$};
    \draw [color=red] (2) -- (4) node {} (-1,-1.5) node [text=red,below] {$b$};
\end{tikzpicture}
\end{center}
\begin {example}
\begin {itemize}
\item $\chi'(C_n)=$
$\begin {cases}
2\:\:\: , 2|n\\
3\:\:\: ,2\not|n\\
\end {cases}$
\begin {center}
\begin {tikzpicture}
\draw 
node[fill=black,circle,inner sep=0pt,minimum size=3pt] (1) at (0,0) {}
node[fill=black,circle,inner sep=0pt,minimum size=3pt] (2) at (0,-1) {}
node[fill=black,circle,inner sep=0pt,minimum size=3pt] (3) at (1,0) {}
node[fill=black,circle,inner sep=0pt,minimum size=3pt] (4) at (1,-1) {}
;
\draw [color=red] (1) -- (2) node {}  ;
\draw [color=red] (3) -- (4) node {}  ;
\draw [color=green] (2) -- (4) node {}  ;
\draw [color=green] (3) -- (1) node {}  ;
\end {tikzpicture}
\begin {tikzpicture}
\draw
node[fill,circle,inner sep=0pt,minimum size=3pt] (n1) at (0,1) {}
node[fill,circle,inner sep=0pt,minimum size=3pt] (n2) at (1.25,0) {}
node[fill,circle,inner sep=0pt,minimum size=3pt] (n3) at (0.75,-1) {}
node[fill,circle,inner sep=0pt,minimum size=3pt] (n4) at (-0.75,-1) {}
node[fill,circle,inner sep=0pt,minimum size=3pt] (n5) at (-1.25,0) {}
;
\draw [color=red] (n1) -- (n2) node {}  ;
\draw [color=red] (n3) -- (n4) node {}  ;
\draw [color=blue] (n5) -- (n4) node {}  ;
\draw [color=green] (n3) -- (n2) node {}  ;
\draw [color=green] (n5) -- (n1) node {}  ;
\end{tikzpicture}
\end{center}
\item $G$ is a bipartite graph, $V(G)=Classes \cup Teachers$.
\newline
There is an edge $tc$ if $t$ is supposed to teach $c$\
\begin {center}
\begin {tikzpicture}
\draw
node[fill,circle,inner sep=0pt,minimum size=3pt] (n1) at (1,0) {}
node[fill,circle,inner sep=0pt,minimum size=3pt] (n2) at (2.5,0){}
node[fill,circle,inner sep=0pt,minimum size=3pt] (n3) at (4,0) {}
node[fill,circle,inner sep=0pt,minimum size=3pt] (n4) at (5.5,0) {}
node[fill,circle,inner sep=0pt,minimum size=3pt] (n5) at (7,0) {}
node[fill,circle,inner sep=0pt,minimum size=3pt] (n6) at (8.5,0) {};
\draw (9,0) node [text=black,right] {classes};
\draw
node[fill,circle,inner sep=0pt,minimum size=3pt] (n7) at (1,-2) {}
node[fill,circle,inner sep=0pt,minimum size=3pt] (n8) at (2.5,-2){}
node[fill,circle,inner sep=0pt,minimum size=3pt] (n9) at (4,-2) {}
node[fill,circle,inner sep=0pt,minimum size=3pt] (n10) at (7,-2) {}
node[fill,circle,inner sep=0pt,minimum size=3pt] (n11) at (8.5,-2) {}
;
\draw (1,-2) node [text=black,below] {Math};
\draw (2.5,-2) node [text=black,below] {Physics};
\draw (4,-2) node [text=black,below] {P.E.};
\draw (7,-2) node [text=black,below] {Geometry};
\draw (8.5,-2) node [text=black,below] {etc.};
\draw (9,-2) node [text=black,right] {teachers};
\draw [color=blue] (n1) -- (n7) node {}  ;
\draw [color=green] (n2) -- (n7) node {}  ;
\draw [color=blue] (n2) -- (n8) node {}  ;
\draw [color=green] (n3) -- (n8) node {}  ;
\draw [color=red] (n3) -- (n7) node {}  ;
\draw [color=red] (n4) -- (n8) node {}  ;
\draw [color=red] (n4) -- (n10) node {}  ;
\draw [color=blue] (n5) -- (n10) node {}  ;
\draw [color=green] (n6) -- (n10) node {}  ;
\end{tikzpicture}
\end{center}
How many time slots do we need to schedule all lessons?
\newline
Answer: $\chi'(G)$. $Colors \equiv$ time slots
\item
$\chi'(K_{12})=11$
\begin {proof}
\begin{minipage}{\textwidth}
Season schedule of the Danish soccer league (figures for the case of $K_6$, $6$ teams)
\begin {center}
\begin {tikzpicture}
\draw
node[fill,circle,inner sep=0pt,minimum size=3pt] (n1) at (0,0) {}
node[fill,circle,inner sep=0pt,minimum size=3pt] (n2) at (-2,-1) {}
node[fill,circle,inner sep=0pt,minimum size=3pt] (n3) at (2,-1) {}
node[fill,circle,inner sep=0pt,minimum size=3pt] (n4) at (2,-2.5) {}
node[fill,circle,inner sep=0pt,minimum size=3pt] (n5) at (-2,-2.5) {}
node[fill,circle,inner sep=0pt,minimum size=3pt] (n6) at (0,-3.5) {};
\draw [color=green] (n1) -- (n3) node {}  ;
\draw [color=purple] (n1) -- (n2) node {}  ;
\draw [color=purple] (n5) -- (n6) node {}  ;
\draw [color=blue] (n1) -- (n4) node {}  ;
\draw [color=red] (n1) -- (n6) node {}  ;
\draw [color=blue] (n3) -- (n6) node {} ;
\draw [color=green] (n3) -- (n5) node {};
\draw [color=green] (n4) -- (n6) node {} ;
\draw [color=red] (n2) -- (n5) node {} ;
\draw [color=red] (n3) -- (n4) node {} ;
\draw [color=blue] (n1) -- (n5) node {}  ;
\end {tikzpicture}
\end {center}
$colors \equiv$ weeks, $color\: classes \equiv$ list of pairs playing that week
\end{minipage}
{\color {purple} first week} {\color {green} second week} {\color {red} third week} etc.
\newline
{\color {blue} (fourth week) \color {black} is not correct because some team is scheduled two games}
\begin {center}
\begin {tikzpicture}
\draw
node[fill,circle,inner sep=0pt,minimum size=3pt] (n1) at (0,0) {}
node[fill,circle,inner sep=0pt,minimum size=3pt] (n2) at (-2,-1) {}
node[fill,circle,inner sep=0pt,minimum size=3pt] (n3) at (2,-1) {}
node[fill,circle,inner sep=0pt,minimum size=3pt] (n4) at (2,-2.5) {}
node[fill,circle,inner sep=0pt,minimum size=3pt] (n5) at (-2,-2.5) {}
node[fill,circle,inner sep=0pt,minimum size=3pt] (n6) at (0,-3.5) {};
\draw [color=green] (n1) -- (n3) node {}  ;
\draw [color=orange] (n3) -- (n4) node {}  ;
\draw [color=green] (n4) -- (n6) node {}  ;
\draw [color=orange] (n6) -- (n5) node {}  ;
\draw [color=green] (n5) -- (n2) node {}  ;
\draw [color=orange] (n2) -- (n1) node {}  ;
\draw [color=red] (n1) -- (n4) node {}  ;
\draw [color=blue] (n1) -- (n6) node {}  ;
\draw [color=blue] (n2) -- (n3) node {}  ;
\draw [color=blue] (n5) -- (n4) node {}  ;
\draw [color=red] (n3) -- (n5) node {}  ;
\draw [color=red] (n2) -- (n6) node {}  ;
\draw [color=yellow] (n1) -- (n5) node {}  ;
\draw [color=yellow] (n2) -- (n4) node {}  ;
\draw [color=yellow] (n3) -- (n6) node {}  ;
\end {tikzpicture}
\end {center}
\end {proof}
\end {itemize}
\end {example}
\begin {lemma}
$\chi'(K_n)=$
$\begin {cases}
n-1\:\:\: , 2|n\\
n\:\:\:, 2\not|n\\
\end {cases}$
\begin {tikzpicture}
\draw
node[fill,circle,inner sep=0pt,minimum size=3pt] (n1) at (0,0) {}
node[fill,circle,inner sep=0pt,minimum size=3pt] (n2) at (2,0) {}
node[fill,circle,inner sep=0pt,minimum size=3pt] (n3) at (4,0) {}
node[fill,circle,inner sep=0pt,minimum size=3pt] (n4) at (6,0) {}
node[fill,circle,inner sep=0pt,minimum size=3pt] (n5) at (5,1) {};
\draw (1,0) node [text=black,below] {$\chi'(K_2)=1$};
\draw (5,0) node [text=black,below] {$\chi'(K_3)=3$};
\draw [color=black] (n1) -- (n2) node {}  ;
\draw [color=blue] (n3) -- (n4) node {};
\draw [color=red] (n5) -- (n3) node {};
\draw [color=green] (n5) -- (n4) node {};
\end {tikzpicture}
\end {lemma}
\begin {proof}
\underline {$n=2k-1$}
Arrange vertices into a regular $n$-gon. Choose a color class consisting of $k$ parallel edges (one vertex is left out). This color class has $n$ possible rotations. These rotations cover all edges of $K_n$. This is an edge-coloring with $n$ colors,so $\chi'(K_n) \leq n$.

Upper bound: take any edge-coloring of $K_n$. The color class 1 must miss some vertex v (because $2\not|n$). Then, $deg(v)=n-1$, and all edges incident to $v$ require $n-1$ colors.Total $\# colors \geq 1+(n-1)=n$
\newline


\underline {$n=2k$}
$deg(v)=n-1$, so we need at least $n-1$ colors, $\chi'(K_n) \geq n-1$. To construct a coloring with $n-1$ colors take the $n-1$ rotations of the following color class: $n-1$ vertices are arranged in a regular $(n-1)$-gon, and one vertex is in the origin. Use $k-1$ parallel edges and one edge from the origin to the vertex on the perimeter being left out.
\end {proof}
\begin {lemma} 
$\chi'(G) \geq \Delta(G)$
\end {lemma}
\begin {proof}
We need at least $\Delta(G)$ colors just to color the edges incident to the vertex of maximum degree.
\end {proof}
\begin {example}
$\Delta(G)=3$, but there is no 3-edge coloring (checked in excercises). There is one with 4 colors, so $\chi'(G)=4$
\end {example}
\begin {center}
\begin {tikzpicture}
\draw
node[fill,circle,inner sep=0pt,minimum size=3pt] (n1) at (0,0) {}
node[fill,circle,inner sep=0pt,minimum size=3pt] (n2) at (2,0) {}
node[fill,circle,inner sep=0pt,minimum size=3pt] (n3) at (0,-2) {}
node[fill,circle,inner sep=0pt,minimum size=3pt] (n4) at (2,-2) {}
node[fill,circle,inner sep=0pt,minimum size=3pt] (n5) at (1,-1) {};
\draw [color=black] (n1) -- (n2) node {}  ;
\draw [color=black] (n2) -- (n4) node {}  ;
\draw [color=black] (n4) -- (n3) node {}  ;
\draw [color=black] (n3) -- (n1) node {}  ;
\draw [color=black] (n5) -- (n2) node {}  ;
\draw [color=black] (n5) -- (n4) node {}  ;
\draw [color=black] (n5) -- (n1) node {}  ;
\end {tikzpicture}
\end {center}
\begin {definition}
A \emph {matching} in G is a set of edges, no two of which have a common endpoint [equivalently, 1-regular (every vertex has deg.1) subgraph of $G$].
\end {definition} 
\begin {observation}\begin{minipage}[t]{\linewidth}
\begin {itemize}
\item If $f$ is an edge-coloring then every color class $f^{-1}(c)$ is a matching.
\item An edge-coloring with $k$ colors is the same as a partition of $E(G)$ into $k$ matchings
\item For any two colors $c_1,c_2\in C,\:c_1\neq c_2$,\:$f^{-1}(c_1)\cup f^{-1}(c_2)$ is a disjoint union of cycles and paths.
\end {itemize}
\end {minipage}
\end {observation}
\begin {center}
%\begin {tikzpicture}
%\draw node[fill,circle,inner sep=0pt,minimum size=3pt] (n1) at (0,0) {}
%	node[fill,circle,inner sep=0pt,minimum size=3pt] (n2) at (-1,-1) {}
%	node[fill,circle,inner sep=0pt,minimum size=3pt] (n3) at (0,-2) {}
%	node[fill,circle,inner sep=0pt,minimum size=3pt] (n4) at (1,-1) {}
%	node[fill,circle,inner sep=0pt,minimum size=3pt] (n5) at (0,-1) {}
%	node[fill,circle,inner sep=0pt,minimum size=0pt] (n6) at (-1.2,-1) {}
%	node[fill,circle,inner sep=0pt,minimum size=0pt] (n7) at (-0.2,-2) {}
%	node[fill,circle,inner sep=0pt,minimum size=0pt] (n8) at (0.2,-2) {}
%	node[fill,circle,inner sep=0pt,minimum size=0pt] (n9) at (0.2,-1.3) {}
%	node[fill,circle,inner sep=0pt,minimum size=0pt] (n10) at (0.4,-1.2) {}
%	node[fill,circle,inner sep=0pt,minimum size=0pt] (n11) at (1.2,-1.2) {}
%	node[fill,circle,inner sep=0pt,minimum size=0pt] (n12) at (1.2,-0.8) {}
%	node[fill,circle,inner sep=0pt,minimum size=0pt] (n13) at (0.2,0.2) {} ;
%\draw [color=blue] (n1) -- (n4) node {}  ;
%\draw [color=green] (n4) -- (n3) node {}  ;
%\draw [color=red] (n3) -- (n2) node {}  ;
%\draw [color=black] (n2) -- (n1) node {}  ;
%\draw [color=green] (n1) -- (n5) node {}  ;
%\draw [color=red] (n5) -- (n4) node {}  ;
%\draw [color=blue] (n5) -- (n3) node {}  ;
% \draw [line width = 1 pt, black, -, ] (n6) edge node {} (n7);
% \draw [line width = 1 pt, black, -, bend right] (n7) edge node {} (n8);
%\draw  [line width = 1 pt, black, -, ] (n8) edge node {} (n9);
% \draw [line width = 1 pt, black, -, bend right] (n7) edge node {} (n8);
% \draw [line width = 1 pt, black, -, bend left] (n9) edge node {} (n10);
% \draw [line width = 1 pt, black, -,] (n10) edge node {} (n11);
% \draw [line width = 1 pt, black, -, bend right] (n11) edge node {} (n12);
% \draw [line width = 1 pt, black, -,] (n12) edge node {} (n13);
%\draw (0.5,0) node [text=black,right] {path};
%\end {tikzpicture}
\begin {tikzpicture}
\draw node[fill,circle,inner sep=0pt,minimum size=3pt] (n1) at (0,0) {}
	node[fill,circle,inner sep=0pt,minimum size=3pt] (n2) at (-1,-1) {}
	node[fill,circle,inner sep=0pt,minimum size=3pt] (n3) at (0,-2) {}
	node[fill,circle,inner sep=0pt,minimum size=3pt] (n4) at (1,-1) {}
	node[fill,circle,inner sep=0pt,minimum size=3pt] (n5) at (0,-1) {}
	node[fill,circle,inner sep=0pt,minimum size=0pt] (n6) at (-1.2,-1) {}
	node[fill,circle,inner sep=0pt,minimum size=0pt] (n7) at (-0.2,-2) {}
	node[fill,circle,inner sep=0pt,minimum size=0pt] (n8) at (0.2,-2) {}
	node[fill,circle,inner sep=0pt,minimum size=0pt] (n9) at (0.2,-1.3) {}
	node[fill,circle,inner sep=0pt,minimum size=0pt] (n10) at (0.4,-1.2) {}
	node[fill,circle,inner sep=0pt,minimum size=0pt] (n11) at (1.2,-1.2) {}
	node[fill,circle,inner sep=0pt,minimum size=0pt] (n12) at (1.2,-0.8) {}
	node[fill,circle,inner sep=0pt,minimum size=0pt] (n13) at (0.2,0.2) {}
	node[fill,circle,inner sep=0pt,minimum size=0pt] (n14) at (-0.2,0.2) {} ;
\draw [color=blue] (n1) -- (n4) node {}  ;
\draw [color=green] (n4) -- (n3) node {}  ;
\draw [color=red] (n3) -- (n2) node {}  ;
\draw [color=black] (n2) -- (n1) node {}  ;
\draw [color=green] (n1) -- (n5) node {}  ;
\draw [color=red] (n5) -- (n4) node {}  ;
\draw [color=blue] (n5) -- (n3) node {}  ;
\draw [line width = 1 pt, black, -, bend right] (n7) edge node {} (n8);
\draw [line width = 1 pt, black, -,] (n8) edge node {} (n11);
\draw [line width = 1 pt, black, -, bend right] (n11) edge node {} (n12);
\draw [line width = 1 pt, black, -, ] (n12) edge node {} (n13);
\draw [line width = 1 pt, black, -, bend right] (n13) edge node {} (n14);
\draw [line width = 1 pt, black, -, ] (n14) edge node {} (n7);
\draw (0.5,0) node [text=black,right] {cycle};
\end {tikzpicture}
\end {center}
\begin {proof}
The first two parts are clear. For part 3, note that in $f^{-1}(c_1)\cup f^{-1}(c_2)$ every vertex is of degree $\leqslant2$.
\end {proof}
\begin {definition}
The \emph {line graph} $G$ is the graph $L(G)$ which keeps track of incidences between edges of $G$.
\newline
Formally:
$V(L(G))=E(G)$ 
\newline
$e_1e_2 \in E(L(G))$ if $e_1,e_2$ share a common vertex in $G$ (here ${e_i\in E(G), e_i\in V(L(G))}$).
\end {definition}
\begin {example}
\begin {tikzpicture}
\draw  
        node[fill=black,circle,inner sep=0pt,minimum size=3pt] (1) at (0,0) {}
        node[fill=black,circle,inner sep=0pt,minimum size=3pt] (2) at (0,-2) {}
        node[fill=black,circle,inner sep=0pt,minimum size=3pt] (3) at (2,0) {}
        node[fill=black,circle,inner sep=0pt,minimum size=3pt] (4) at (-2,-1) {}
        node[fill=black,circle,inner sep=0pt,minimum size=3pt] (5) at (2,-2) {}
        node[fill=red,circle,inner sep=0pt,minimum size=3pt] (6) at (6,0) {}
       node[fill=red,circle,inner sep=0pt,minimum size=3pt] (7) at (8,0) {}
       node[fill=green,circle,inner sep=0pt,minimum size=3pt] (8) at (6,-2) {}
         node[fill=blue,circle,inner sep=0pt,minimum size=3pt] (9) at (8,-2) {}
   node[fill=blue,circle,inner sep=0pt,minimum size=3pt] (10) at (7,-1) {}
       node[fill=green,circle,inner sep=0pt,minimum size=3pt] (11) at (7,1) {}
  ; 
    \draw [color=blue] (1) -- (2) node {} (0,-1) node [text=blue,right] {$c$};
    \draw [color=red] (1) -- (3) node {} (1,0) node [text=red,above] {$d$};
    \draw [color=green] (1) -- (4) node {} (-1.5,-0.75) node [text=green,above] {$a$};
    \draw [color=blue] (3) -- (5) node {} (2,-1) node [text=blue,right] {$e$};
    \draw [color=green] (5) -- (2) node {} (1,-2.2) node [text=green,below] {$f$};
    \draw [color=red] (2) -- (4) node {} (-1.5,-1.5) node [text=red,below] {$b$};
    \draw (-2.3,-1) node [text=black,left] {$G=$};
\draw [color=black] (7) -- (9) node {};
\draw [color=black] (9) -- (8) node {};
\draw [color=black] (8) -- (6) node {};
\draw [color=black] (6) -- (10) node {};
\draw [color=black] (8) -- (10) node {};
\draw [color=black] (11) -- (10) node {};
\draw [color=black] (11) -- (6) node {};
\draw [color=black] (11) -- (7) node {};
\draw [color=black] (10) -- (7) node {};
	\draw (5.5,-1) node [text=black,left] {$L(G)=$};
\draw (7,1) node [text=black,above] {a};
\draw (8.1,0) node [text=black,right] {d};
\draw (5.8,0) node [text=black,left] {b};
\draw (5.8,-2) node [text=black,left] {f};
\draw (8.2,-2) node [text=black,right] {e};
\draw (7,-1.2) node [text=black,below] {$c$};
\end{tikzpicture}
\end {example}
\bigskip
\textbf{\underline {Glossary}}
\newline
\newline
{edge in $G\equiv$} {vertex of $L(G)$}
\newline
{matching in $G \equiv$} {independent set in {$L(G)$}
\newline
{edge-coloring of $G\equiv$} {vertex coloring of {$L(G)$}
\newline
{$\chi'(G)$={$\chi(L(G))$}
\begin {observation}
$\omega(L(G))\geq \Delta(G)$ because all edges incedent to a fixed vertex $v$, induce a clique in $L(G)$.That implies
\newline
$\chi'(G)=\chi(L(G))\geq \omega(L(G))\geq \Delta(G)$ which we already knew. Moreover:
\newline
$\Delta(L(G))= \max_{uv\in E(G)}(deg_G(u)+deg_G(v)-2)\leq2\Delta(G)-2$
\newline
\begin {center}
\begin {tikzpicture}
\draw
	 node[fill=black,circle,inner sep=0pt,minimum size=3pt] (1) at (0,0) {}
	  node[fill=black,circle,inner sep=0pt,minimum size=3pt] (2) at (4,0) {}
	   node[fill=black,circle,inner sep=0pt,minimum size=0pt] (3) at (-1.5,0.75) {}
	    node[fill=black,circle,inner sep=0pt,minimum size=0pt] (4) at (-1.5,0.25) {}
	     node[fill=black,circle,inner sep=0pt,minimum size=0pt] (5) at (-1.5,-0.25) {}
	      node[fill=black,circle,inner sep=0pt,minimum size=0pt] (6) at (-1.5,-0.75) {}
	       node[fill=black,circle,inner sep=0pt,minimum size=0pt] (7) at (5.5,0.75) {}
	        node[fill=black,circle,inner sep=0pt,minimum size=0pt] (8) at (5.5,0.25) {}
	         node[fill=black,circle,inner sep=0pt,minimum size=0pt] (9) at (5.5,-0.25) {}
	          node[fill=black,circle,inner sep=0pt,minimum size=0pt] (10) at (5.5,-0.75) {};
	          \draw [color=black] (1) -- (2) node {};
	          \draw [color=black] (1) -- (3) node {};
	          \draw [color=black] (1) -- (4) node {};
	          \draw [color=black] (1) -- (5) node {};
	          \draw [color=black] (1) -- (6) node {};
	          \draw [color=black] (2) -- (7) node {};
	          \draw [color=black] (2) -- (8) node {};
	          \draw [color=black] (2) -- (9) node {};
	          \draw [color=black] (2) -- (10) node {};
	          \draw (0,0) node [text=black,above] {u};
	          \draw (4,0) node [text=black,above] {v};
	          \draw (2,-0.2) node [text=black,below] {e};
	          \draw (-1.5,-0) node [text=black,left] {$deg(u)-1$};
	          \draw (5.5,0) node [text=black,right] {$deg(v)-1$};
\end {tikzpicture}
\end {center}
Now, greedy coloring gives:
\newline
$\chi'(G)=\chi(L(G)) \leq \Delta(L(G))+1\leq 2\Delta(G)-1$
\newline
We proved: $\Delta(G)\leq \chi'(G) \leq 2\Delta(G)-1$
\end {observation}
\begin {theorem}
(Vizing '64) For any graph $G$ we have
$\Delta(G)\leq \chi'(G) \leq \Delta(G)+1$.
\end {theorem}
\begin {remark}
It is NP-hard to recognize if $\chi'(G)=\Delta(G)$ or $\chi'(G)=\Delta(G)+1$, even for graphs with $\Delta(G)=3$.
\newline
Graphs with $\chi'(G)=\Delta(G)$ are called $Class\ 1$
\newline
Graphs with $\chi'(G)=\Delta(G)+1$ are called $Class\ 2$
\newline
But we can still indentify some graph classes for which $\chi'(G)=\Delta(G)$.
\end {remark}
\begin {theorem}
(K\"onig) If $G$ is bipartite then $\chi'(G)=\Delta(G)$
\end {theorem}

\underline {Hall's marriage theorem}(1935-original version): We have a number of boys and girls, each boy fancies some of the girls. Is it possible to arrange marriages, so that each boy marries some girls he likes?
\newline
Obvious necessary condition: "Each set of $k$ boys ($k\geq 1$) likes, altogether, at least $k$ different girls".
\begin {theorem}
(Hall) If $G$ is a bipartite graph with parts $V(G)=A\cup B$, such that for every $X\subseteq A$ we have:
$$|\bigcup_{x\in X}N_G(x)|\geq |X|$$
then $G$ has a matching of size $|A|$.
\end {theorem}
\newpage
\begin {center}
\begin {tikzpicture}
\draw (5,0) ellipse (4cm and 0.5cm);
\draw
	 node[fill=black,circle,inner sep=0pt,minimum size=3pt] (1) at (2,0) {};
	 \draw
	 node[fill=black,circle,inner sep=0pt,minimum size=3pt] (2) at (5,0) {};
	 \draw
	 node[fill=black,circle,inner sep=0pt,minimum size=3pt] (3) at (5.5,0) {};
	 \draw
	 node[fill=black,circle,inner sep=0pt,minimum size=3pt] (4) at (6,0) {};
	 \draw
	 node[fill=black,circle,inner sep=0pt,minimum size=3pt] (5) at (6.5,0) {};
	 \draw
	 node[fill=black,circle,inner sep=0pt,minimum size=3pt] (6) at (7,0) {};
	 \draw (1,0) node [text=black,left] {$A$};
	 \draw (2,0.5) node [text=black,above] {$x$};
	  \draw (6.5,0.8) node [text=black,left] {$X$};
	   \draw (9.5,0) node [text=black,right] {$boys$};
\draw (6,0) ellipse (1.3cm and 0.2cm);
	  
\draw (5,-2) ellipse (4cm and 0.5cm);
\draw
	 node[fill=black,circle,inner sep=0pt,minimum size=3pt] (7) at (2,-2) {};
	 \draw
	 node[fill=black,circle,inner sep=0pt,minimum size=3pt] (8) at (2.5,-2) {};
	 \draw
	 node[fill=black,circle,inner sep=0pt,minimum size=3pt] (9) at (3,-2) {};
\draw [color=black] (1) -- (7) node {};
\draw [color=black] (1) -- (8) node {};
\draw [color=black] (1) -- (9) node {};
 \draw (1,-2) node [text=black,left] {$B$};
 \draw (2,-2.3) node [text=black,below] {$N(x)$};
 \draw (5.5,-2) node [text=black] {\bf{$\geqslant|X|$}};
 \draw (9.5,-2) node [text=black,right] {$girls$};
\draw (2.4,-2) ellipse (0.8cm and 0.2cm);
\draw  node[fill=black,circle,inner sep=0pt,minimum size=0pt] (10) at (7,-2) {};
\draw
	 node[fill=black,circle,inner sep=0pt,minimum size=0pt] (11) at (6.6,-2) {};
\draw
	 node[fill=black,circle,inner sep=0pt,minimum size=0pt] (12) at (6.2,-2) {};
\draw
	 node[fill=black,circle,inner sep=0pt,minimum size=0pt] (13) at (5.8,-2) {};
\draw  node[fill=black,circle,inner sep=0pt,minimum size=0pt] (14) at (5.4,-2) {};
\draw  node[fill=black,circle,inner sep=0pt,minimum size=0pt] (15) at (5,-2) {};
\draw  node[fill=black,circle,inner sep=0pt,minimum size=0pt] (16) at (4.6,-2) {};
\draw [color=black] (2) -- (13) node {};
\draw [color=black] (3) -- (16) node {};
\draw [color=black] (4) -- (12) node {};
\draw [color=black] (4) -- (15) node {};
\draw [color=black] (5) -- (14) node {};
\draw [color=black] (5) -- (11) node {};
\draw [color=black] (6) -- (10) node {};
\draw (5.7,-2) ellipse (1.7cm and 0.4cm);
\end{tikzpicture}
\end {center}
\begin {proof}
Write $N(X)=N_G(X)=\bigcup_{x\in X} N_G(x)$. Induction on $n=|A|$
\newline
\tikz \node[circle,draw] {1}; If $n=1$
\begin {tikzpicture}
\draw (0,0) ellipse (1cm and 0.3cm);
\draw  node[fill=black,circle,inner sep=0pt,minimum size=3pt] (1) at (0.3,0) {};
 \draw (1,0) node [text=black,right] {$A$};
\draw (1,-1) ellipse (1cm and 0.3cm);
\draw  node[fill=black,circle,inner sep=0pt,minimum size=3pt] (2) at (1.3,-1) {};
 \draw (1.9,-1) node [text=black,right] {$B$};
  \draw (2.8,-1) node [text=black,right] {the unique vertex of $A$ has at least one edge.};
 \draw [color=black] (1) -- (2) node {};
\end {tikzpicture}
\newline
\newline
\tikz \node[circle,draw] {2a}; For any $X\subsetneq A$, $|N_G(X)|\geq|X|+1$.
\newline
Then take any $a\in A$, $b\in N_G(a)$, match them and use induction on $G[A-a,B-b]$.
\newline
\newline
\begin {tikzpicture}
\draw  node[fill=black,circle,inner sep=0pt,minimum size=0pt] (1) at (0,0) {};
\draw  node[fill=black,circle,inner sep=0pt,minimum size=0pt] (2) at (3,0) {};
\draw  node[fill=black,circle,inner sep=0pt,minimum size=0pt] (3) at (3,-2) {};
\draw  node[fill=black,circle,inner sep=0pt,minimum size=0pt] (4) at (0,-2) {};
\draw  node[fill=black,circle,inner sep=0pt,minimum size=3pt] (5) at (3.6,-0.5) {};
\draw  node[fill=black,circle,inner sep=0pt,minimum size=3pt] (6) at (3.6,-1.5) {};
\draw [color=black] (1) -- (2) node {};
\draw [color=black] (2) -- (3) node {};
\draw [color=black] (3) -- (4) node {};
\draw [color=black] (4) -- (1) node {};
\draw (1.5,0) node [text=black,above] {$G[A-a,B-b]$};
\draw (3.6,-0.5) node [text=black,right] {$a$};
\draw (3.6,-1.5) node [text=black,right] {$b$};
\draw (2.6,-1.5) ellipse (2cm and 0.3cm);
\draw (4.5,-1.5) node [text=black,right] {$B$};
\draw (4.5,-0.5) node [text=black,right] {$A$};
\draw (2.6,-0.5) ellipse (2cm and 0.3cm);
\end {tikzpicture}

Possible because for any $X\subseteq A-a$ we have 
$|N(X)|\geq |X|+1-1=|X|$
\newline
\tikz \node[circle,draw] {2b}; $N_G(X)=|X|$ for some $X\subsetneq A$.
\newline
then: by induction find a matching in $G[X,N(X)]$
\newline
Also, there is a matching in $G[A-X,B-N(X)]$
\newline
Possible,because for any $Y\subseteq A-X$
\newline
\begin{tikzpicture}
\draw (0,0) ellipse (2cm and 0.4cm);
\draw (0,-1) ellipse (2cm and 0.4cm);
\draw (-1.5,0) node [text=black,] {$X$};
\draw (1.2,0) node [text=black,] {$A-X \ \color{red}Y$};
\draw (-1.5,-1) node [text=black,] {$N(X)$};
\draw (1,-1) node [text=black,] {$B-N(X)$};
\draw  node[fill=black,circle,inner sep=0pt,minimum size=0pt] (1) at (-1.5,-0.8) {};
\draw  node[fill=black,circle,inner sep=0pt,minimum size=0pt] (2) at (1.6,0) {};
\draw  node[fill=black,circle,inner sep=0pt,minimum size=0pt] (3) at (1.7,0) {};
\draw  node[fill=black,circle,inner sep=0pt,minimum size=0pt] (4) at (1.8,0) {};
\draw  node[fill=black,circle,inner sep=0pt,minimum size=0pt] (5) at (1,-0.8) {};
\draw  node[fill=black,circle,inner sep=0pt,minimum size=0pt] (6) at (1.1,-0.8) {};
\draw  node[fill=black,circle,inner sep=0pt,minimum size=0pt] (7) at (1.2,-0.8) {};
\draw  node[fill=black,circle,inner sep=0pt,minimum size=0pt] (8) at (1.3,-0.8) {};
\draw [color=red] (1) -- (2) node {};
\draw [color=red] (3) -- (5) node {};
\draw [color=red] (4) -- (7) node {};
\draw [color=red] (2) -- (8) node {};
\end {tikzpicture}
\newline
$|Y|+|X|=|Y\cup X|\leqslant|N_G(Y\cup X)|=|N(X)|+|N_G(Y)\cap(B-N(X))|=|X|+|N_G(Y)\cap(B-N(X))|$
\newline
So $|Y|\leqslant |N_G(Y)\cap(B-N(X))|$ and therefore induction applies to $G[A-X,B-N(X)]$
\end {proof}
\end{document}