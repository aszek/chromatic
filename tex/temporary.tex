\chapter{Temporary}

\section{Notation}

A graph is $G=(V,E)$ and it has $n=|V|$, $m=|E|$. If there are more graphs the next one is $H$. A coloring with $c$ colors is a function $f:V\to \{1,\ldots,c\}=C$. The chromatic number is $\chi$. A bipartite graph has parts $A,B$, or $X,Y$.


\section{Aha foo bar}
This is just a testing ground for now.

Here is ade finition

\begin{definition} We say that a \emph{definition} is a definition is
\begin{equation}
\label{eq:def-eq-temp}
\int_M d\omega = \int_{\partial M} \omega
\end{equation}
\end{definition}

On the other hand, here is a remark:

\begin{remark}
I would like the contents of a remark not to be italicised.
\end{remark}

\section{Section}
It would be good to split each chapter into 2-4 sections.

\section{Including sage code}
We include SAGE source code like this:

\begin{verbatim}
def FunnyGraph(n):
    c = graphs.CompleteGraph(n)
    c.delete_edges(graphs.CycleGraph(n).edges())
    return graphs.MycielskiStep(c).join(graphs.WheelGraph(n+1))

G = FunnyGraph(99)
\end{verbatim}

At the very end I will implement it using a pygmentize highlighter for python.

Figures can be drawn in tikz or whatever, or included from another file. In any case, it would be good to have every figure inside a figure environment with a label and caption.

\section{TODOs}
List of todos for MA:
\begin{itemize}
\item Remove this file
\item Add pygmentize
\item Expand and check bibliography
\end{itemize}


\section{Notation}
(This section is not temporary. In fact, it will become this whole chapter.)

\begin{center}
\begin{tabular}{ll}
$G, H, \ldots$          	&  graphs \\
$G=(V,E)$					&  vertices and edges of a graph \\
$\compl{G}$					&  graph complement \\
$L(G)$						&  line graph of $G$ \\
$G\setminus e$				&  edge removal \\
$G/e$						&  edge contraction \\
$K_n$						&  $n$-vertex complete graph \\
$K_{n,m}$					&  complete bipartite graph \\
$C_n$						&  $n$-vertex cycle \\
$P_n$						&  $n$-vertex path \\
$Q_d$						&  the $d$-dimensional cube graph \\
$\omega(G)$					&  clique number \\
$\alpha(G)$					&  independence number \\
$\chi(G)$					&  chromatic number \\
$\chi_l(G)$					&  list chromatic number \\
$\chi'(G)$					&  edge chromatic number \\
$P_G(t), P(G,t)$			&  chromatic polynomial \\
$G(n,p)$					&  random graph $G(n,p)$ \\
$G+H$						&  join of graphs \\
$G\sqcup H$					&  disjoint union of graphs
\end{tabular}
\end{center}