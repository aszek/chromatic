\chapter{Preface}

These lecture notes come from a master-level course \emph{Graph coloring}, taught by the first author at the Department of Mathematical Sciences of the University of Copenhagen in spring 2016.

\medskip
Despite the name, this is not what one would call a comprehensive course in graph coloring. Instead, graph coloring problems serve only as a convenient excuse to familiarize the students, who may never have taken a more advanced combinatorics course, with interesting combinatorial techniques. The purpose is rather to give a taste of tools from linear algebra, calculus, combinatorics and geometry and demonstrate a few classical combinatorial theorems and their applications. It also has a bit of an experimental flavour and includes short programming exercises in Sage.

\medskip
Each chapter ends with a series of exercises, preferably to be solved with the guidance of an instructor. The last chapter contains exam problems with short hints. This division does not imply a gradation of difficulty --- some of the chapter exercises are quite challenging!

\medskip
We are grateful for the contribution by the students who took notes during the course: Giorgia Laura Cassis, Hugr\'un Fj\'ola Hafsteinsd\'ottir, Rolf J{\o}rgensen, Mathis Elmgaard Isaksen, Sokratis Theodoridis, Mortan Janusarson Thomsen, Kristoffer Holm Nielsen. Most of these notes are based on their writeup.

\medskip
The LaTeX source of these lecture notes is available under the GNU GPL license from \url{http://github.com/aszek/chromatic}. 