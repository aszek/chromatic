\chapter{Vertex coloring}

\section{Definitions and examples}

Goal: Assign colours to the vertices of a graph, such that adjacent vertices have different colours (the problem is optimized by finding the minimal number of colours needed for this process).

\begin{example}
\begin{enumerate}
\item Colouring a land-map, such that the countries that share a border have different colours. In this case the vertices represent the countries and we draw an edge whenever two countries share a border.
\item Scheduling the timetable for an exam session. In this case the vertices represent the different subjects offered and we draw an edge whenever there is at least a student following both courses. Minimizing the number of colours means programming the minimal number of exam-spots, such that each student can take all the exams he/she applied for.
\end{enumerate}
\end{example}

\begin{example} Take a graph $G$ with $81$ vertices forming a $9\times9$ grid, such that the following conditions are satisfied:
\begin{itemize}
\item Every row is a clique,
\item Every column is a clique,
\item If we divide the grid forming $9$ $3\times 3$ subgrids, each of these is a clique.
\end{itemize}
Then each possible colouring represents the solutions of a Sudoku.
\end{example}

\begin{definition} A \emph{(vertex-)colouring} of $G$ with colour set $C$ is a function
$$c:V(G)\longrightarrow C,$$
such that if $xy\in E(G)$, then $c(x)\neq c(y)$.
\end{definition}

\begin{definition} The \emph{chromatic number} $\chi(G)$ is the smallest number $k$ for which there is a colouring of $G$ with $k$ colours. If $\chi(G)\leqslant k$, then $G$ is called \emph{$k$-colourable}.
\end{definition}

\begin{definition} Each $c^{-1}(i)$ is called a \emph{color class}.
\end{definition}

\begin{example}
\begin{itemize}
\item $\chi(P_n)=2$,
\item $\chi(C_n)=
\begin{cases} 
2, & \text{if } 2\mid n\\ 
3, & \text{if } 2\not| \ n
\end{cases},$
\item $\chi(K_n)=n$,
\item $\chi(Q_n)=2$, since $Q_n$ is bipartite. Moreover, $\chi$ of every bipartite graph with at least one edge is $2$.
\end{itemize}
\end{example}

\begin{lemma}
\begin{enumerate}
\item $\chi(G)\leqslant |V(G)|$,
\item $\chi(G)=|V(G)| \Longleftrightarrow G$ is complete,
\item $\chi(G)=1 \Longleftrightarrow G \simeq \overline{K_n},$
\item $\chi(G)=2 \Longleftrightarrow G$ is bipartite and has at least one edge.
\end{enumerate}
\end{lemma}

\begin{proof}
\begin{enumerate}
\item $c: |V(G)| \longrightarrow  |V(G)|$ with $c(v)=v$ is a colouring with exactly $|V(G)|$ colours.
\item One direction is obvious. Let us assume $G$ is not complete, then we can find $x,y\in V(G)$ with $xy\notin E(G)$. Set $c(x)=c(y)$ and colour all the other $|V(G)|-2$ vertices with different colours: this is a colouring with $|V(G)|-1$ colours.
\item One direction is obvious. For the other look at $c:V(G)\longrightarrow \{1\}$ with $c(x)=c(y)$ for all $x,y\in V(G)$. This implies $xy\notin E(G)$ for all $x,y\in V(G)$, hence $G \simeq \overline{K_n}$.
\item One direction is obvious. For the other, if $\chi(G)=2$ then it suffices to define $A:=c^{-1}(1)$ and $B:=c^{-1}(2)$ to find the partition of the set $V(G)$.
\end{enumerate}
\end{proof}

\begin{remark}
Already determining whether $\chi(G)\leqslant 3$ or $\chi(G)\geqslant 4$ is hard.
\end{remark}

\begin{observation}
Every color class is an independent set.
\end{observation}
\begin{proof}
If $c(x) = c(y)$ then $xy \notin E(G)$.
\end{proof}

\begin{lemma}
The following are equivalent:
\begin{enumerate}[(1)]
\item $G$ is $k$-colorable
\item $V(G)$ can be partioned into $k$ independent sets
\item There exists a graph homomorphism $G \rightarrow K_k$
\end{enumerate}
\end{lemma}
\begin{proof}
(1) $\Leftrightarrow$ (2): Assume (1) and choose a coloring $c \colon G
\rightarrow \{1,\ldots,k\}$. Then $c^{-1}(1),\ldots,c^{-1}(k)$
partitions $G$ into $k$ independent sets. On the other hand if (2) holds
we can write $V(G) = V_1 \cup \cdots \cup V_k$ where each $V_j$ is an
independet set and $V_i \cap V_j = \emptyset$ when $i \neq j$. Define a
coloring $c \colon G \rightarrow \{1,\ldots,k\}$ by $c(v) = i$ if $v \in
V_i$. Then $c$ is clearly a coloring. So we have that (1)
$\Leftrightarrow$ (2).

(2) $\Leftrightarrow$ (3): If $f \colon G \rightarrow K_k$ is a
homomorphism then $f^{-1}(i)$ is an independent set. This shows (3)
$\Rightarrow$ (2). Suppose $V(G) = V_1 \cup \cdots \cup V_k$, $V_i \cap
V_j = \emptyset$, with each $V_i$ an independent set. Define $f \colon G
\rightarrow K_k$ by $f(v) = i$ if $v \in V_i$. If $vw \in E(G)$ then $v
\in V_i$ and $w \in V_j$ for $i \neq j$. Then $f(v)f(w) = ij \in E(K_k)$
since $i \neq j$.
\end{proof}

The next lemma says that ``bigger graphs have bigger chromatic
numbers''.
\begin{lemma}
If $H \subset G$ ($H$ is a subgraph of $G$) then $\chi(H) \le \chi(G)$.
\end{lemma}
\begin{proof}
Any coloring $c \colon V(G) \rightarrow C$ restricts to a coloring $c
\colon V(H) \rightarrow C$.
\end{proof}


\begin{lemma}
For any graph $G$
\begin{enumerate}[(1)]
\item $\chi(G) \ge \omega(G)$
\item $\chi(G) \ge \frac{|V(G)|}{\alpha(G)}$
\end{enumerate}
\end{lemma}
\begin{proof}
(1): If $\omega(G) = k$, then $G$ contains a clique on $k$ vertices,
i.e. $K_k \subset G$. Then $k = \chi(K_k) \le \chi(G)$ by the lemma.

(2): Suppose $\chi(G) = k$. Then some color class has size at least
$\frac{1}{k}|V(G)|$. Therefore we have an independent set of size $\ge
\frac{1}{k}|V(G)|$, which means $\alpha(G) \ge \frac{1}{k}|V(G)|$.
\end{proof}

\begin{example}
$\chi(C_{2n + 1}) = 3$ although $\omega(C_{2n + 1}) = 2$ for $n \ge 2$.
So it can happen that $\chi > \omega$.
\end{example}

\section{Greedy algorithm(s)}
\begin{definition}
\emph{The greedy coloring} relative to a vertex ordering $v_1, \ldots
v_n$ of the vertices of $G$ is obtained by coloring in this order
subject to the rule:
\[
c(v_i) = \text{ first available color that does not appear among }
N(v_i) \cap \{v_1,\ldots v_{i-1}\}
\]
\end{definition}

\begin{theorem}
The greedy algorithm uses at most $\Delta(G) + 1$ colors. In particular
$\chi(G) \le \Delta(G) + 1$ for any $G$.
\end{theorem}
\begin{proof}
When coloring $v_i$ at most $\Delta(G)$ colors are forbidden, because
$v_i$ has at most $\Delta(G)$ neighbors, so it has $\le \Delta(G)$
already colored neighbors. They use $\le \Delta(G)$ different colors, so
at least one color from $\{1,\ldots,\Delta(G) + 1\}$ is available for
$v_i$.
\end{proof}
\begin{observation}
There is room for improvement if we choose some special vertex ordering.
\end{observation}
\begin{example}
Order the vertices so that $\deg(v_1) \ge \deg(v_2) \ge \cdots \ge
\deg(v_n)$.
\end{example}
\begin{theorem}
With respect to the above ordering, the greedy method uses at most $1 +
\max_i(\min(\deg(v_i),i-1))$ colors. In particular $\chi(G) \le 1 +
\max_i\min(\deg(v_i),i-1)$.
\end{theorem}
\begin{proof}
$v_i$ has at most $\min(\deg(v_i),i-1)$ neighbors among $v_1,\ldots,v_{i-1}$.
\end{proof}


\begin{example}
Order the $v_i's$ so that for each $i$, $v_i$ is some vertex of smallest
degree in $G[\{v_1,\ldots,v_i\}]$ (the sub-graph induced by
$\{v_1,\ldots,v_i\}$).
Construction:
\begin{align*}
v_n &= \text{ vertex of minimal degree in } G \\
v_{n-1} &= \text{ vertex of minimal degree in } G-v_n \\
v_{n-2} &= \text{ vertex of minimal degree in } G-v_n-v_{n-1} \\
\vdots & 
\end{align*}
\end{example}

\begin{theorem}
With this ordering, the greedy method needs at most $1 + \max_H \delta(H)$ colors, 
where the maximum is over all induced subgraphs $H$ of $G$.
\end{theorem}
\begin{proof}
$v_i$ has exactly $\deg_{G[\{v_1,\ldots,v_i\}]}(v_i)$ neighbors among
$v_1,\ldots,v_{i-1}$. But that number is exactly
$\delta(G[\{v_1,\ldots,v_i\}]) \le \max_H \delta(H)$
\end{proof}


\section{Brooks' theorem}

\begin{example}
Where $\chi = \Delta + 1$:
\begin{enumerate}[(1)]
\item $\chi(K_n) = n = \Delta(K_n) + 1$
\item $\chi(C_{2n + 1}) = 3 = \Delta(C_{2n+1}) + 1$
\end{enumerate}
\end{example}

\begin{theorem}[Brooks]
If $G$ is a connected graph, which is not complete, nor an odd cycle,
then $\chi(G) \le \Delta(G)$.
\end{theorem}
\begin{proof}
Let $\Delta = \Delta(G)$. If $\Delta \le 2$ then $G$ is a path or a
cycle, and the result is easy. Assume $\Delta \ge 3$. Pick any vertex $v
\in V(G)$ with $\deg(v) = \Delta$. Let $H = G-v$.  The proof is by
induction on the number of vertices $|V(G)|$. Suppose, for
contradiction, $G$ is not $\Delta$-colorable. $H$ is a graph with
$\Delta(H) \le \Delta$. By induction (since $|V(H)| = |V(G)| - 1$) $H$
is $\Delta(H)$-colorable so in particular $\Delta$-colorable. (Be
careful: what if $H$ is an odd cycle or a complete graph? We dealt with these cases in the exercises). Fix any
coloring $c$ of $H$ with $\Delta$ colors. Since $\deg(v) = \Delta$ we can
assume that all neighbors of $V$ have different colors. Otherwise there is a
spare color we can assign to $v$ and $G$ would be $\Delta$-colorable.
Label the vertices in $N(v)$ such that $N(v) =
\{v_1,\ldots,v_{\Delta}\}$ and $c(v_i) = i$. Define $H_{i,j}$ be the
subgraph of $H$ induced by vertices of colors $i$ and $j$ ($i \neq j$,
$i,j \in \{1,\ldots,\Delta\}$).
\begin{claim}[0]
$H_{i,j}$ is bipartite and $v_i,v_j \in V(H_{i,j})$.
\end{claim}
\begin{proof}
Obvious.
\end{proof}
\begin{claim}[1]
$v_i,v_j$ are in the same connected component $C_{i,j}$ of $H_{i,j}$
\end{claim}
\begin{proof}
If not then $H_{i,j}$ is the disjoint union of two subgraphs: $H_{i,j} =
H_{i,j}^{(1)} \cup H_{i,j}^{(1)}$ such that $v_i \in
H_{i,j}^{(1)}$ and $v_j \in H_{i,j}^{(2)}$. Define a new coloring $c'$ of $H$
by swapping the colors in $H_{i,j}^{(2)}$ i.e. define $c'(v) = i$ if $c(v) = j$
and vice versa for all $v \in H_{i,j}^{(2)}$. (The new $c'$ is still a
coloring). With this new coloring $v_i$ and $v_j$ have the same colors,
so we have a spare color for $v$ which as before is a contradiction.
\end{proof}
\begin{claim}[2]
$C_{i,j}$ is a path.
\end{claim}
\begin{proof}
First note that $\deg_{H_{i,j}}(v_i) = 1$ because otherwise $v_i$ has
two neighbors of color $j$. Moreover $\deg_H(v_i) \le \Delta - 1$. Then
there exist $k \neq i$ such that we can recolor $v_i$ with color $k$ and
still have a coloring of $H$ and we get a spare color for $v$.
Analogously we have $\deg_{H_{i,j}}(v_j) = 1$. Let $u$ be a degree $\ge
3$ vertex in $C_{i,j}$ closest to $v_i$. Then $c(u) = i$ or $c(u) = j$,
assume $c(u) = i$. Again: $\deg_H(u) \le \Delta$, but neighbors of $u$
have at most $\Delta - 2$ colors. We can recolor $u$ with some $k
\neq i,j$. Then $v_i,v_j$ land in different components of $H_{i,j}$
contrary to claim 1. 
\end{proof}
\begin{claim}[3]
$C_{i,j} \cap C_{i,k} = \{v_i\}$ if $i \neq j \neq k \neq i$.
\end{claim}
\begin{proof}
If not then there is some $u \in C_{i,j} \cap C_{i,k}$ such that $u \neq
v_i$. $\deg(u) \le \Delta$ and $u$ has two neighbors of color $j$ and
two of color $k$. Therefore neighbors of $u$ use $\le \Delta - 2$ colors
and we can recolor $u$ with some color $\neq i,j,k$ and get a proper
coloring. The new coloring contradicts claim 1.
\end{proof}
If all of $v_1,\ldots,v_{\Delta}$ are pairwise adjacent then $G =
K_{\Delta + 1}$, or otherwise $\Delta(G) \ge \Delta + 1$.  So, assume without
loss of generality that $v_1v_2 \notin E(G)$. Consider the paths
$C_{1,2},C_{1,3}$. Let $C_{1,2} = v_1u\cdots v_2$. Flip the colors in
$C_{1,3}$. We get a proper coloring of $G$. In this new coloring
$C_{1,2}$ and $C_{3,2}$ both contain $u$ contradicting claim 3.
\end{proof}



\section{Large-$\chi$ triangle-free graphs deterministically and probabilistically}

We know that $\chi(G)\geq\omega(G)$ \\
If $G=C_{2n+1}, \quad n\geq 2$ then $\omega(G)=2,\quad \chi(G)=3$ \\
Q: Is there a graph with $\omega(G)=2$ and $\chi(G)=4$? \\
A: Yes the smallest such is the Gr{\"o}tszch graph on 11 vertices $G_{11}$ \\

It is not possible to bound $\chi(G)$ in terms of $\omega(G)$.
\begin{theorem} 
For any $k\geq2$ there is a triangle-free graph with $\chi(G)=k$
\end{theorem}
\begin{definition}
Suppose $G$ is a graph with $V(G)=\{v_1,..,v_n\}$. The \emph{Mycielski construction} $M(G)$ is a new graph with $V(M(G))=\{v_1 ,..,v_n\}\cup\{u_1,..,u_n\}\cup\{w\}, \quad E(M(G))=\{wu_i,\, i=1,..,n\} \cup \{v_i v_j,\ u_iv_j\ : v_iv_j \in E(G)\}$.

(FIGURE)
\end{definition}

As an example have that $M(K_2)=C_5$ and $M(C_5)=G_{11}$.

\begin{theorem} (restated) If $G$ is triangle-free and $\chi(G)=k$ then $M(G)$ is triangle free and $\chi(M(G))=k+1$.
\end {theorem}
\begin{proof}
\begin{enumerate}
	\item $M(G)$ is still triangle free \\
	The only possibility is a triangle with $1$ vertex from $U$ and 2 vertices from $V$. However, by the definition of $M(G)$ we then have that $v_iv_jv_k$ is a triangle, contradicting that $G$ was triangle-free.
	\item $M(G)$ is $k+1$ colorable \\
	We can color $G$ with $k$ colors (The set $V$ in the picture). We use another color $(k+1)$ for all vertices in $U$, and the last node $w$ is colored with some color different from $(k+1)$ 
	\item $M(G)$ is not $k$-colorable. \\
	Suppose otherwise, $c:V(M(G)) \rightarrow \{1,..,k\}$. Suppose wlog that $w$ has color $k$, then $U$ is colored with $k-1$ colors. (Our goal is to show that we can color $G$ with $k-1$ colors) \\
	Let $A= \{v_i \in V, \, c(v_i)=k\}$. \\
	Recolor $A$ by changing the color of each $v_i$ to $c(u_i)$ \\
	$$
	c'(v_i)=\left\{ \begin{array}{rl}
c(v_i) &\mbox{ if $c(v_i)\neq k$} \\
c(u_i) &\mbox{ if $c(v_i)=k$}
\end{array} \right.
$$
$C'$ uses only colors $ \{1,..,k-1\} $. Let's check that $c'$ is a coloring of $G$. Suppose $c'(v_i)=c'(v_j)$ 
\begin{enumerate}
	\item If both of these $v_i,v_j \notin A$ then $v_iv_j \notin E(G)$  
	\item If $v_i \in A,\, v_j \notin A$. Suppose $v_iv_j \in E(G)$, so that $u_iv_j \in E(M(G))$

(FIGURE)

Then $c(u_i)=c'(v_i)=c'(v_j)=c(v_j)$ Contradicting that $c$ is a coloring of $M(G)$
	\item If $v_i,v_j \in A$. Then $c(v_i)=c(v_j)=k \Rightarrow v_iv_j \notin E(G)$
\end{enumerate}
\end{enumerate}
\end{proof}

Recap: We can have triangle-free graphs with arbitrarily high $\chi$. But is $M(G)$ just a special construction that achieves this? Not really. We will see that in the following:

Goal: Prove that objects with an interesting property $P$ exists by showing that a random object has $P$ with non-zero probability. For example: $P=$"triangle-free and $\chi(G)\geq k$"

\begin{definition} (construction). Fix $V=\{1,..,n\}, \, 0\leq p\leq 1$. Construct a graph on $V$ by taking each edge $ij$, $0 \leq i <j \leq n$ with probability $p$, independently for each each pair. \\
$G(n,p)$ is the probability space of all graphs obtained in this way.

\end{definition}

A graph $G \in G(n,p)$ has probability
$$
\mathbb{P}(G)=p^{|E(G)|}(1-p)^{\binom{n}{2}-|E(G)|}.
$$
The expected number of edges/triangles in a random graph from $G(n,p)$ is
$$
\mathbb{E}[\# \text{edges}]=\binom{n}{2}\mathbb{P}[ij \text{ is an edge}]=p\binom{n}{2}
$$
$$
\mathbb{E}[\# \text{of triangles in } G(n,p)]=\binom{n}{3}\mathbb{P}[ijk \text{ is a triangle}]=\binom{n}{3}p^3
$$
Sage can generate random graphs from $G(n,p)$ (\texttt{graphs.randomGNP(n, p)}). 
For the next part we need a few prerequisites:
\begin{enumerate}
	\item $1-x \leq e^{-x}$,
	\item $\binom{n}{k} \leq n^k$,
	\item Markov's inequality: if $X$ is a non-negative random variable, $t>0$, then $\mathbb{P}[X>t]\leq \frac{1}{t}\mathbb{E}[X]$.
\end{enumerate}
\begin{theorem} For every $k\geq 2$  there is a triangle-free graph with $\chi(G)\geq k$.
\end{theorem}
Remark: Not only "there is" but "there are many".
\begin{proof}
Take $G \in G(n,p)$ where $p=\frac{1}{n^{5/6}}$. Let $X=\#$ of triangles in $G$. \\
$\mathbb{E}[X]=\binom{n}{3}p^3 \leq n^{1/2}$. By Markov's inequality $\mathbb{P}[X>10n^{1/2}]\leq \frac{1}{10}$, which means that a typical $G$ has very few triangles. \\
To show that $\chi(G)$ is "large" we will prove that $\alpha(G)$ is "small". Let $a=\frac{3}{p}\ln{n}$ 
$$
\mathbb{P}[\alpha(G)\geq a] \leq \binom{n}{a}\mathbb{P}[\text{exists independent set of size }a] = \binom{n}{a}(1-p)^{\binom{a}{2}} \leq n^ae^{-p\binom{a}{2}} \rightarrow 0,\quad n\rightarrow \infty
$$
where the last limit can be computed by plugging in the formulae for $p$ and $a$ in terms of $n$.

For sufficiently large $n$ we have $\mathbb{P}[\alpha(G)\geq a]<\frac{1}{10}$. Then $$\mathbb{P}[X<10\sqrt{n} \text{ and } \alpha(G)<a] \geq \frac{8}{10}.$$

We showed, with probability $\geq \frac{8}{10}$, a random graph from $G(n,p), \, p=\frac{1}{n^{5/6}}$, has $<10\sqrt{n}$ triangles, and $\alpha(G)<a<3n^{5/6}\ln{n}$ \\
Completing the proof: \\
Take such a random $G$. Remove at most $10\sqrt{n}$ vertices so we get a triangle free graph $G'$ \\
$|V(G')| \geq \frac{n}{2}$ (for large $n$) and 
$$
\chi(G')\geq \frac{|V(G')|}{\alpha(G')} \geq \frac{n/2}{3n^{5/6}\ln{n}}=\frac{1}{6}\frac{n^{1/6}}{\ln{n}}\rightarrow \infty, \quad n\rightarrow \infty
$$
So $\chi(G')$ can be arbitrarily large
\end{proof}


\section{Basic constructions}
\begin{enumerate}
	\item Disjoint union $G \sqcup H$. $\chi(G \sqcup H)=\max(\chi(G),\chi(H))$. As $G\subseteq G \sqcup H \supseteq H$, we need at least enough colors to color $H,G$ individually.
	\item Wedge $G \vee H$ Graphs joined at a single vertex.

(FIGURE)

$\chi(G\vee H)=\max(\chi(G),\chi(H)$. After coloring $G$ color $H$, perhaps permuting the colors so that they agree on the common vertex..	
	
\item The sum (join) $G+H$. It is the disjoint union together with all edges between $V(G)$ and $V(H)$ \\
$\chi(G+H)=\chi(G)+\chi(H)$. As the colors of $G$ must be different from the colors of $H$.
\item The cartesian product $G \square H$ \\
$V(G \square H) = V(G) \times V(H)$ \\
$\{(u,v),(u',v')\} \text{ is an edge iff } (u=u' \text{ and } vv' \in E(H)) \text{ or } (uu' \in E(G) \text{ and } v=v')$. 

\begin{lemma} $\chi(G \square H)=\max(\chi(G),\chi(H))$
\end{lemma}
\begin{proof}
$G\subseteq G \square H \supseteq H$ so $\chi(G \square H) \geq \max(\chi(G),\chi(H))$. Suppose $G$ and $H$ both have coloring with color set $\{0,..,k-1\}$ ($k=\max(\chi(G), \chi(H))$). Le these colorings be 
$$
f: V(G) \rightarrow \{ 0,..,k-1\},
$$
$$
f': V(H) \rightarrow \{ 0,..,k-1 \}.
$$
Define $F(u,v)=f(u)+f'(v) \mod k$. We will check that $F$ is a coloring. Let $(u,v)(u',v') \in E(G \square H)$. Wlog let $u=u'$ and $vv' \in E(H)$, then:
\begin{align*}
	F(u,v) &=f(u)+f'(v) \mod k \\
	F(u',v') &=f(u')+f'(v') \mod k \\
	         &= f(u)+f'(v') \mod k \\
					&\neq f(u)+f'(v) \mod k  \quad \text{because } vv' \in E(H) \\
					&=F(u,v).
\end{align*}
\end{proof}

\end{enumerate}




\section{Exercises}

\begin{enumerate}
\item Prove that a graph with chromatic number $k$ has at least ${k\choose 2}$ edges.
\item Prove $\chi(G)\leq |V(G)|-\alpha(G)+1$.
\item Suppose we have $n$ straight lines in the plane, no three intersecting at the same point. Let $G$ be the graph whose vertices are the intersection points of the lines, and two of them are adjacent if they appear consecutively on one of the lines. Prove that $G$ is $3$-colorable.
\item Suppose $G$ is $4$-colorable. Prove that $G$ can be written as a union of two bipartite graphs (that means there are subgraphs $H_1,H_2$ of $G$ such that $V(G)=V(H_1)\cup V(H_2)$, $E(G)=E(H_1)\cup E(H_2)$ and $H_1,H_2$ are bipartite).
\item Prove that if $\overline{G}$ is bipartite then $\chi(G)=\omega(G)$.
\item Prove that if an $n$-vertex graph is $k$-colorable then it has at most $\frac{k-1}{2k}n^2$ edges.
\item Show that every graph has a vertex ordering for which the greedy algorithm uses exactly $\chi(G)$ colors.

\item $\max_i\min(\deg(v_i),i-1) \le \Delta(G)$ so this bound is at least as
good as $\Delta(G) + 1$ (often much better).

\item 
A graph is called $k$-critical if $\chi(G)=k$ and $\chi(H)<k$ for every proper subgraph $H\subseteq G$.

\begin{enumerate}
\item Prove that a graph is $3$-critical if and only if it is an odd cycle.
\item Show that a $k$-critical graph $G$ has $\delta(G)\geq k-1$.
\item Suppose $G$ is $k$-critical and $e\in E(G)$. Show that every $(k-1)$-coloring of $G-e$ assigns different colors to the endpoints of $e$.
\item Prove that $C_5+K_k$ is $(3+k)$-critical.
\end{enumerate}

\item Show that $\chi(G)\geq h$ iff $\alpha(G \square K_k) \geq |V(G)|$.
\end{enumerate}