\documentclass[a4paper]{article}
\usepackage[utf8]{inputenc}
\usepackage[english]{babel}
\usepackage{color}              %Farben, f.r \definecolor{}
\usepackage{amssymb}            %Mathematische Symbole
\usepackage{amsthm}             %Besseres \newtheorem
\usepackage{amsmath}           %Mathematische Umgebungen
\usepackage{mathtools}          %\xRightarrow, etc
\usepackage{mathrsfs}           %enthaelt \mathscr
\usepackage{graphicx}
\usepackage{enumerate}          % in-place numerations def.
\usepackage{fullpage}

\usepackage{array}
%\usepackage{multicol}
%\usepackage[notref,notcite]{showkeys}
%\usepackage{algorithm,algorithmic}
\usepackage{color}

\usepackage{graphicx}
\usepackage{xypic}
\entrymodifiers={+!!<0pt,\fontdimen22\textfont2>}
\usepackage[all]{xy}

\newtheoremstyle{myremark} % name
    {7pt}                    % Space above
    {7pt}                    % Space below
    {}  	                 % Body font
    {}                           % Indent amount
    {\bf}       	         % Theorem head font
    {.}                          % Punctuation after theorem head
    {.5em}                       % Space after theorem head
    {}  % Theorem head spec (can be left empty, meaning ‘normal’)

\theoremstyle{plain}
\newtheorem{lemma}{Lemma}
\newtheorem{theorem}[lemma]{Theorem}
\newtheorem{fact}[lemma]{Fact}
\newtheorem{definition}[lemma]{Definition}
\newtheorem{corollary}[lemma]{Corollary}
\newtheorem{proposition}[lemma]{Proposition}
\newtheorem{conjecture}[lemma]{Conjecture}
\newtheorem{observation}[lemma]{Observation}
\newtheorem{problem}[lemma]{Problem}
\newtheorem{notation}[lemma]{Notation}
\newtheorem*{claim}{Claim}

\theoremstyle{myremark}
\newtheorem{remark}[lemma]{Remark}
\newtheorem{example}[lemma]{Example}
\newtheorem{exercise}[lemma]{Exercise}

%%%%%% EDIT HERE: %%%%%%%%%%%
\newcommand{\LECTURENUMBER}{11}
\newcommand{\LECTURETITLE}{Vizing's Theorem. Chromatic Number of
$\mathbb{R}^k$.}
\newcommand{\LECTURESCRIBE}{Rolf C. Jørgensen}
\newcommand{\codt}{\cdot}
\newcommand{\codts}{\cdots}

%% Dokument Beginn %%%%%%%%%%%%%%%%%%%%%%%%%%%%%%%%%%%%%%%%%%%%%%%%%%%%%%%%
\begin{document}
\thispagestyle{empty}

\begin{center}
	{\Large\bf Graph coloring}\\
	{\bf Lecture notes, vol. \LECTURENUMBER, \LECTURETITLE}\\
\end{center}
Lecturer: Michal Adamaszek \hfill Scribe: \LECTURESCRIBE
\begin{center}
\line(1,0){450}
\end{center}

%%%%%%% EDIT ALSO BELOW: %%%%%%%%%%%%%%%%

\begin{theorem} (Vizing)
For every graph $G$:
\[
\Delta(G) \le \chi'(G) \le \Delta(G) + 1.
\]
\begin{proof} (Sketch)
Let $\Delta := \Delta(G)$. We have to show that there is an edge coloring with
$\Delta + 1$ colors.

If $G = \overline{K_n}$ we are done.

Otherwise choose an edge $e = uv \in E(G)$. Fix $C = \{1, \ldots, \Delta
+ 1\}$. By induction ($\Delta(G-e) \le \Delta$) color the edges of $G -
e$. Let $f \colon E(G) \setminus \{e\} \rightarrow C$ denote the
coloring. We need to define a color of $e$, possibly changing some
existing colors. We say a color $c$:
\begin{enumerate}
\item \emph{appears at a vertex $x$} if $f(xy) = c$ for some $xy \in
E(G)$. 
\item \emph{is missing at $x$}, otherwise.
\end{enumerate}
Since $|C| = \Delta + 1$, each vertex has at
least one missing color. Define $v_0 := v$ and set $c_0$ to be any color
missing at $v_0$. If $c_0$ is missing at $u$, then set $f(uv_0) := c_0$
and we are done. Otherwise if $c_0$ appears at $u$, then set $v_1$ to be
a vertex such that $f(uv_1) = c_0$ and $c_1$ any color missing at $v_1$.
If $c_1$ is missing at $u$ shift the colors from $uv_1$ to $uv_0$. If
$c_1$ appears at $u$ then choose a vertex $v_2$ such that $f(uv_2) =
c_1$ and any color $c_2$ missing at $v_2$. Recursively we define $v_i$
as any vertex with $f(uv_i) = c_{i-1}$ and $c_i$ as any missing color at
$v_i$. This process stops when either 
\begin{enumerate}
\item $c_i$ is also missing at $u$, then shift colors from $uv_i$ to
$uv_0$.
\item $c_i = c_j$ for $0 \le j < i$.
\end{enumerate}
Suppose $c_i = c_j$ for $0 \le j < i$. Let $c$ be some color missing at
$u$. If $c$ is also missing at $v_i$, set $f(uv_i) = c$ and shift colors
$uv_i$ to $uv_0$, otherwise $c$ appears at $v_i$. Consider the graph
$f^{-1}(c) \cup f^{-1}(c_i)$. It contains a path starting at $v_i$.
Where does it end? If the path ends at $v_j$, shift colors from $uv_j$
down to $uv_0$ set $f(uv_j) = c$ and flip the colors on the path $v_i
\rightarrow v_j$. Otherwise it ends at $v_{j+1}$ or somewhere else, and these two cases are left as an exercise.
\end{proof}
\end{theorem}
\begin{remark}
We only relied on the existence of missing colors at every vertex. We
can use this observation, for example:
\end{remark}
\begin{proposition}
If $G$ has only one vertex of maximal degree, then $\chi'(G) =
\Delta(G)$.
\begin{proof}
Let $u$ have $\deg(u) = \Delta = \Delta(G)$. Pick an edge $e = uv \in E(G)$. Now
$\Delta(G - e) \le \Delta - 1$. Color the edges of $G - e$ with $\Delta$
colors (Vizing). Again every vertex has a missing color. The recoloring
part of the proof gives now an edge coloring of $G$ with $\Delta$
colors.
\end{proof}
\end{proposition}

\hrule
\section*{Chromatic number of the Euclidean spaces}
In this part we will have infinite graphs.
\begin{definition}
$\chi(\mathbb{R}^d)$ is the minimal number of colors required to color
all points in $\mathbb{R}^d$ so that if $d(x,y) = 1$ then $x,y$ have
different colors for all $x,y \in \mathbb{R}^d$, where 
\[
d(x,y) = \sqrt{\sum_i (x_i - y_i)^2}
\]
\end{definition}
\begin{definition}
For $X \subset \mathbb{R}^d$ define a graph $U_X$ (U for ``unit'') with
vertex set $X$ and edges
\[
x_1x_2 \in E(U_X) \text{ iff } d(x_1,x_2) = 1.
\]
\end{definition}
\begin{observation}
$\chi(\mathbb{R}^d) = \chi(U_{\mathbb{R}^d})$.
\end{observation}
\begin{example}
$U_{\mathbb{R}}$: $xy \in E(U_{\mathbb{R}})$ iff $|x - y| = 1$.
$U_{\mathbb{R}}$ is a union of infinitely many (uncountably many)
bi-infinite paths. $\chi(U_{\mathbb{R}}) = 2 = \chi(\mathbb{R})$.
\end{example}
\begin{remark}
All invariants $(\omega, \chi, \alpha, \Delta, \ldots)$ we defined still
make sense for infinite graphs, except that they might be equal to
$\infty$. $\omega(G) \le \chi(G)$ and $H \subset G \Rightarrow \chi(H)
\le \chi(G)$ etc. still hold.
\end{remark}

\begin{theorem}
Suppose $G$ is a graph (which may be infinite).  If every \emph{finite}
subgraph of $G$ can be colored with $k$ colors, then $G$ can be colored with $k$ colors.
\begin{proof}
Let $G = (V,E)$ be a graph and let $X$ be the set of all functions $f
\colon V \rightarrow \{1,\ldots,k\}$, i.e. $X = \prod_{v \in
V}\{1,\ldots,k\} = \{1,\ldots,k\}^V$. View $\{1,\ldots,k\}$ as a
discrete topological space and equip $X$ with the product topology.
$\{1,\ldots,k\}$ is finite, so it is compact. By Tychonoff's theorem $X$
is compact. For any $F \subset E$ let $X_F \subset X$ be defined as
those $f \colon V \rightarrow \{1,\ldots,k\}$ which are proper colorings
of $(V,F)$.
\begin{enumerate}
\item $X_{\{e\}}$ is closed in $X$ since
\[
X_{\{e\}} = \bigcup_{i\neq j}\{f \in X : f(u) = i, f(v) = j, e = uv \}
\]
is a finite union of closed sets.
\item $X_{F_1} \cap X_{F_2} = X_{F_1 \cup F_2}$.
\item For any $F \subset E$, $X_F$ is closed since $X_F = \bigcap_{e \in
F}X_{\{e\}}$, is an intersection of closed sets, hence closed.
\end{enumerate}
Now: Take the family $\mathcal{F} = \{X_F\}_{\stackrel{F \subset E}{F
\text{ finite}}}$. All sets in $\mathcal{F}$ are closed, and all
intersections of finitely many from $\mathcal{F}$ are non-empty (second
claim: $X_{F_1} \cap \codts \cap X_{F_n} = X_{F_1 \cup \codts \cup F_n}
\neq \emptyset$ because $(V, F_1 \cup \cdots \cup F_n)$ is finite, hence
$k$-colorable) Then the intersection of all sets in $\mathcal{F}$ is
non-empty (by compactnes of $X$). $f \in \bigcap_{\stackrel{ F \subset
E}{|F| < \infty }}X_F$ is a proper coloring on \emph{every} edge of $G$.
\end{proof}
\end{theorem}

\subsection*{What about $\chi(\mathbb{R}^2)$}
\begin{lemma}
(easy upper-bound) $\chi(\mathbb{R}^2) \le 9$.
\begin{proof}
Take the $3 \times 3$-square where the length of the diagonals in each
little square is $0.99$. Color every such square with $9$ colors (choose
any neighboring color on the commom edges). Use this square to tile the
plane. Take two points $x,y$ of the same color. Then
\begin{enumerate}
\item $x,y$ are in the same small square and so $d(x,y) \le 0.99$, or
\item $x,y$ are in two different big squares and $d(x,y) \ge 2 \cdot
0.99 \cdot 1/\sqrt{2} > 1$
\end{enumerate}
so $d(x,y) \neq 1$.
\end{proof}
\end{lemma}


%%%%%%%%%%%%%%%%%%%%%%%%%%%%%%%%%%%%%%%%
\begin{thebibliography}{9}
\bibitem{west} West, \textit{Introduction to graph theory}

\end{thebibliography}

\end{document}




