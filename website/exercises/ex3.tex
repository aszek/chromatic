%
\documentclass[a4paper]{article}


\usepackage{color}              %Farben, f.r \definecolor{}
\usepackage{amssymb}            %Mathematische Symbole
\usepackage{amsthm}             %Besseres \newtheorem
\usepackage{amsmath}           %Mathematische Umgebungen
\usepackage{mathtools}          %\xRightarrow, etc
\usepackage{mathrsfs}           %enthaelt \mathscr
\usepackage[matrix,arrow,curve]{xy}     %Diagramme
\usepackage{graphicx}
\usepackage{enumerate}          % in-place numerations def.
\usepackage{fullpage}

%% Dokument Beginn %%%%%%%%%%%%%%%%%%%%%%%%%%%%%%%%%%%%%%%%%%%%%%%%%%%%%%%%
\begin{document}
\pagestyle{empty}
\begin{center}
	{\Large\bf Graph coloring}\\
	{\large\bf Exercise class problems - volume 3}\\
	\line(1,0){330}
\end{center}

\begin{enumerate}
\item Prove that the Petersen graph is not planar by finding an iterated subdivision of $K_{3,3}$ or $K_5$ as a subgraph.
\item 
\begin{itemize}
\item Prove that any planar graph has a vertex of degree at most $5$.
\item Use it to deduce the \textbf{6-color theorem}: Every planar graph is $6$-colorable.
\end{itemize}
\item How many edges must be removed from the Petersen graph to make the result planar?
\item Prove that $\overline{Q_3}$ is not planar.
\item Prove that $Q_4$ is not planar.
\item Prove that $v-e+f=k+1$ in a planar graph with $k$ connected components.
\item Prove (without invoking any difficult theorems) that any triangle-free planar graph is $4$-colorable.
\item Suppose $G$ is planar. Show that we can assign two colors to the vertices of $G$ so that there are no monochromatic triangles. (A monochromatic triangle is a triangle in $G$ with all vertices of the same color) You may use the $4$-color theorem.
\end{enumerate}

\begin{center}
	\line(1,0){330}
\end{center}
	
A graph is called $k$-critical if $\chi(G)=k$ and $\chi(H)<k$ for every proper subgraph $H\subseteq G$.

\begin{enumerate}
\item Prove that a graph is $3$-critical if and only if it is an odd cycle.
\item Show that a $k$-critical graph $G$ has $\delta(G)\geq k-1$.
\item Suppose $G$ is $k$-critical and $e\in E(G)$. Show that every $(k-1)$-coloring of $G-e$ assigns different colors to the endpoints of $e$.
\item Prove that $C_5+K_k$ is $(3+k)$-critical.
\end{enumerate}
	
\end{document}




