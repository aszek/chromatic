\documentclass[a4paper]{article}
\usepackage{color}              %Farben, f.r \definecolor{}
\usepackage{amssymb}            %Mathematische Symbole
\usepackage{amsthm}             %Besseres \newtheorem
\usepackage{amsmath}           %Mathematische Umgebungen
\usepackage{mathtools}          %\xRightarrow, etc
\usepackage{mathrsfs}           %enthaelt \mathscr
\usepackage{graphicx}
\usepackage{enumerate}          % in-place numerations def.
\usepackage{fullpage}
\usepackage{hyperref}

\usepackage{array}
%\usepackage{multicol}
%\usepackage[notref,notcite]{showkeys}
%\usepackage{algorithm,algorithmic}
\usepackage{color}

\usepackage{graphicx}
\usepackage{xypic}
\entrymodifiers={+!!<0pt,\fontdimen22\textfont2>}
\usepackage[all]{xy}

\usepackage{float}
\usepackage{tikz}
\usepackage{tikz-cd}
\usepackage{tikz,fullpage}
\usetikzlibrary{arrows,%
                petri,%
                topaths}%
\usepackage{tkz-berge}
\usepackage[position=top]{subfig}
\usetikzlibrary{shapes.geometric}
\usetikzlibrary{decorations.markings}

\newtheoremstyle{myremark} % name
    {7pt}                    % Space above
    {7pt}                    % Space below
    {}  	                 % Body font
    {}                           % Indent amount
    {\bf}       	         % Theorem head font
    {.}                          % Punctuation after theorem head
    {.5em}                       % Space after theorem head
    {}  % Theorem head spec (can be left empty, meaning ‘normal’)

\theoremstyle{plain}
\newtheorem{lemma}{Lemma}
\newtheorem{theorem}[lemma]{Theorem}
\newtheorem{fact}[lemma]{Fact}
\newtheorem{definition}[lemma]{Definition}
\newtheorem{corollary}[lemma]{Corollary}
\newtheorem{proposition}[lemma]{Proposition}
\newtheorem{conjecture}[lemma]{Conjecture}
\newtheorem{observation}[lemma]{Observation}
\newtheorem{problem}[lemma]{Problem}
\newtheorem{notation}[lemma]{Notation}
\newtheorem*{claim}{Claim}

\theoremstyle{myremark}
\newtheorem{remark}[lemma]{Remark}
\newtheorem{example}[lemma]{Example}
\newtheorem{exercise}[lemma]{Exercise}
\newtheorem{algorithm}[lemma]{Algorithm}
\newtheorem{application}[lemma]{Application}
\newtheorem*{goal}{Goal}

\newcommand{\oo}[1]{\overline{#1}}
\newcommand{\RR}{\mathbb{R}}
\newcommand{\ZZ}{\mathbb{Z}}
\newcommand{\vol}{\mathrm{Vol}}

%%%%%% EDIT HERE: %%%%%%%%%%%
\newcommand{\LECTURENUMBER}{12}
\newcommand{\LECTURETITLE}{Chromatic numbers of cube-like graphs.}
\newcommand{\LECTURESCRIBE}{M.A.}

%% Dokument Beginn %%%%%%%%%%%%%%%%%%%%%%%%%%%%%%%%%%%%%%%%%%%%%%%%%%%%%%%%
\begin{document}
\thispagestyle{empty}

\begin{center}
	{\Large\bf Graph coloring}\\
	{\bf Lecture notes, vol. \LECTURENUMBER \\ 
	\LECTURETITLE}\\
\end{center}
Lecturer: Michal Adamaszek \hfill Scribe: \LECTURESCRIBE
\begin{center}
\line(1,0){450}
\end{center}

%%%%%%% EDIT ALSO BELOW: %%%%%%%%%%%%%%%%

We are about to prove an exponential lower bound $\chi(\RR^d)\geq c_1^d$ on the chromatic number of $\RR^d$. To this end we introduced modified cube graphs $Q_d(u)$ with vertices $\oo{x}=(x_1,\ldots,x_d)$, $x_i\in\{0,1\}$ and edges between $\oo{x}$ and $\oo{y}$ whenever $\oo{x}$ and $\oo{y}$ differ in exactly $u$ places. (Throughout we will use the overbar $\oo{x}$ to denote vectors). In the natural geometric embedding of the cube these edges all have the same Euclidean length $\sqrt{u}$, therefore $Q_d(u)$ are unit distance graphs in $\RR^d$ and $\chi(\RR^d)\geq \chi(Q_d(u))$.


The graphs $Q_d(u)$ give pretty good lower bounds on $\chi(\RR^d)$ already for small $d$. Here are results which can be verified using the Sage code we wrote in the exercises:
\begin{itemize}
\item $\chi(Q_5(2))=8$. Consequently, $\chi(\RR^5)\geq 8$. The best known lower bound is $9$.
\item $\alpha(Q_{10}(4))=40$ (this will take about 20min in Sage). Consequently
$$\chi(\RR^{10})\geq\chi(Q_{10}(4))\geq\frac{|V(Q_{10}(4))|}{\alpha(Q_{10}(4))}=\frac{2^{10}}{40}=25.6,$$
that is $\chi(\RR^d)\geq 26$. This is the best known bound!
\end{itemize}

In order to prove some lower bounds valid for all $d$ we need to add a further complication to $Q_d(u)$.

\begin{definition}
The graph $Q_d(u,s)\subseteq Q_d(u)$ is the subgraph of $Q_d(u)$ induced by the vertices with exactly $s$ coordinates equal to $1$. Precisely:
$$V(Q_d(u,s))=\{\oo{x}=(x_1,\ldots,x_d)~:~x_i\in\{0,1\},\ \sum_{i=1}^d x_i=s\}$$
and $\oo{x}$ and $\oo{y}$ are adjacent in $Q_d(u,s)$ iff they differ in exactly $u$ positions.
\end{definition}

\begin{example}
$Q_3(2,1)$ has vertex set $\{001,010,100\}$ and it is isomorphic to $K_3$.
\end{example}

As in the computational examples above, it is usually easier to say something about the independence number $\alpha$ than directly about the chromatic number $\chi$. Our main theorem, which we will prove in the next part of the lecture, is the following.

\begin{theorem}
If $p$ is a prime then 
$$\alpha(Q_d(2p,2p-1))\leq {d\choose 0}+{d\choose 1}+\cdots+{d\choose p-1}.$$
\end{theorem}

We will prove this theorem in a moment. Let us just note that the condition ``p is a prime'' suggests that this fact is somewhat algebraic in nature. For now, let us see what this theorem buys us when it comes to chromatic numbers.

\begin{theorem}
We have $\chi(\RR^d)\geq 1.05^d$ for sufficiently large $d$.
\end{theorem}
\begin{proof}
For any prime $p\leq d/2$ we have
$$\chi(\RR^d)\geq\chi(Q_d(2p))\geq\chi(Q_d(2p,2p-1))\geq
\frac{|V(Q_d(2p,2p-1))|}{\alpha(Q_d(2p,2p-1))}\geq \frac{{d\choose 2p-1}}{p{d\choose p-1}}$$
where in the last step we used the inequality of Theorem 3 and the observation $|V(Q_d(u,s))|={d\choose s}$.

Intuitively, the last fraction will be maximized if the binomial coefficient ${d\choose 2p-1}$ is close to the middle of the $d$-th row of the Pascal triangle, that is when $p\approx d/4$. Since we can only use $p$ primes, we resort to a classical number-theoretic result of Czebyschev: every interval $[n,2n]$ contains a prime. That allows us to choose a prime $p$ such that $\frac{d}{8}\leq p\leq \frac{d}{4}$. By carefully cancelling common factors in the binomial coefficients we obtain:

$$\chi(\RR^d)\geq \frac{1}{p}\cdot\frac{d-p+1}{2p-1}\cdot\frac{d-p}{2p-2}\cdots\frac{d-2p+2}{p}.$$
Under the condition $d\geq 4p$ each of the last $p$ factors is $\geq \frac{3}{2}$, so:
$$\chi(\RR^d)\geq\frac{1}{p}\Big(\frac{3}{2}\Big)^p\geq\frac{4}{d}\Big(\Big(\frac{3}{2}\Big)^{\frac{1}{8}}\Big)^d\geq \frac{4}{d}\cdot 1.051^d\geq 1.05^d$$
where the last inequality holds for sufficiently large $d$.
\end{proof}

\section*{Proof of Theorem 3}
Before jumping to the proof, let us review two combinatorial methods of proving inequalities like $A\leq B$, where $A, B$ are some combinatorially defined quantities.

\smallskip
\noindent
\textbf{Method 1 --- set comparison}. If a set of size $B$ contains a subset of size $A$ then $A\leq B$.

\begin{example} We will show that ${n\choose k}\leq 2^n$. The family of all subsets of $\{1,\ldots,n\}$ has size $2^n$, and it contains the family of all $k$-element subsets, the latter of size ${n\choose k}$. Our inequality follows.
\end{example}

That was an easy and completely standard argument. Our next method is also based on an elementary observation in linear algebra.

\smallskip
\noindent
\textbf{Method 2 --- vector space comparison}. If a vector space of dimension $B$ contains $A$ linearly independent vectors then $A\leq B$.

\smallskip
This may seem like an overkill, but it is actually a  useful strategy in many otherwise complicated situations (like our Theorem 3). Here is an example of how the method works: the (rather classical) problem known as Odd--Town.

\begin{example}
$n$ people participate in $m$ clubs. Every club has an odd number of members, and every two clubs have an even number of common members. Prove that $m\leq n$.

First let's note that we may have $m=n$, for example when every person forms its own one-element club.

To solve the problem, encode the clubs $C_1,\ldots,C_m$ via ``membership vectors'' $\oo{c_1},\ldots,\oo{c_m}$ of length $n$, where
$$(\oo{c_i})_j=
\begin{cases}
1 & \textrm{if person}\ j\ \textrm{belongs to club}\ i,\\
0 & \textrm{otherwise},
\end{cases}
$$
for $i=1,\ldots,m$, $j=1,\ldots,n$. If we write $\langle\oo{x},\oo{y}\rangle=\sum_ix_iy_i$ for the standard inner product, then
\begin{align*}
\langle \oo{c_i},\oo{c_k}\rangle &= \textrm{number of common members of}\ C_i\ \textrm{and}\ C_k,\\
\langle \oo{c_i},\oo{c_i}\rangle &= \textrm{number of members of }\ C_i.
\end{align*}
We will show that $\oo{c_1},\ldots,\oo{c_m}$ are linearly independent. Suppose, for a contradiction, that it is not true. Then we have a linear relation
$$\sum_ia_i\oo{c_i}=0$$
where not all $a_i$ are zero. Since the coordinates of $\oo{c_i}$ are integers, we can assume that all $a_i\in \ZZ$ and moreover $\textrm{gcd}(a_1,\ldots,a_m)=1$. In particular, $a_k$ is odd for some $k$. Now:
$$0=\langle\sum_ia_i\oo{c_i},\oo{c_k}\rangle=a_k\langle \oo{c_k},\oo{c_k}\rangle+\sum_{i\neq k}a_i\langle \oo{c_i},\oo{c_k}\rangle$$
which is a contradiction, because $a_k\langle\oo{c_k},\oo{c_k}\rangle$ is odd, while all the other terms are even.

We showed that $\oo{c_1},\ldots,\oo{c_m}$ are linearly independent vectors in $\RR^n$. It follows that $m\leq n$.
\end{example}

Very similar arguments will now appear in the proof Theorem 3.

\begin{proof}[Proof of Theorem 3]
As always, we write $\langle \oo{x},\oo{y}\rangle=\sum_{i=1}^dx_iy_i$. Let $\oo{x}$ and $\oo{y}$ be two different vertices of $Q_d(2p,2p-1)$. Using the fact that both $\oo{x}$ and $\oo{y}$ have exactly $2p-1$ coordinates equal to $1$, we easily get 
$$|\{j~:~x_j\neq y_j\}|=2(2p-1-|\{j~:~x_j=y_j=1\}|)=2(2p-1-\langle\oo{x},\oo{y}\rangle),$$
hence
$$\langle \oo{x},\oo{y}\rangle=2p-1-\frac12|\{j~:~x_j\neq y_j\}|.$$
Now if $\oo{x}$ and $\oo{y}$ are adjacent in $Q_d(2p,2p-1)$ then they differ in exactly $2p$ places, and we get $\langle\oo{x},\oo{y}\rangle=2p-1-p=p-1$. Otherwise we get some other inner product between $0$ and $2p-2$ (because $\oo{x}\neq\oo{y})$. The upshot is that
$$
\langle\oo{x},\oo{y}\rangle
\begin{cases}
=p-1 & \textrm{if}\ \oo{x}\oo{y}\in E(Q_d(2p,2p-1)),\\
\not\equiv p-1 \pmod{p} &  \textrm{if}\ \oo{x}\oo{y}\not\in E(Q_d(2p,2p-1)).
\end{cases}
$$
Moreover $\langle \oo{x},\oo{x}\rangle=2p-1$ for all $\oo{x}$.

\smallskip
Take any independent set $I$ in $Q_d(2p,2p-1)$. For any $\oo{x}\in I$ consider the function $f_{\oo{x}}:\{0,1\}^d\to\RR$ defined for $\oo{t}=(t_1,\ldots,t_d)$ by the formula
$$f_{\oo{x}}(\oo{t}) = \langle\oo{x},\oo{t}\rangle^{\underline{p-1}}$$
(recall that $z^{\underline{p-1}}=z(z-1)\cdots(z-(p-2))$ is the falling factorial).
The functions $f_{\oo{x}}$ are naturally elements of the $\RR$-vector space of all functions $\{0,1\}^d\to\RR$. Let us check that the set $\{f_{\oo{x}}\}_{\oo{x}\in I}$ is linearly independent in that space. If not, then we would have a linear relation
$$\sum_{\oo{x}\in I}a_{\oo{x}}f_{\oo{x}}=0$$
for $a_{\oo{x}}$ not all zero. As in the example before, we can assume that $a_{\oo{x}}\in\ZZ$ and $\textrm{gcd}(a_{\oo{x}})=1$. In particular, some $a_{\oo{x_0}}$ is not divisible by $p$. We have
$$0=\sum_{\oo{x}\in I}a_{\oo{x}}f_{\oo{x}}(\oo{x_0})=
a_{\oo{x_0}}\langle\oo{x_0},\oo{x_0}\rangle^{\underline{p-1}}+
\sum_{I\ni \oo{x}\neq\oo{x_0}}a_{\oo{x}}\langle\oo{x},\oo{x_0}\rangle^{\underline{p-1}}.
$$

We have $\langle\oo{x_0},\oo{x_0}\rangle^{\underline{p-1}}=(2p-1)(2p-2)\cdots(p+1)\neq 0 \pmod{p}$. Here we use that $p$ is a prime! Since $I$ is an independent set, each $\langle\oo{x},\oo{x_0}\rangle$ is different from $p-1 \pmod{p}$, hence one of the factors in the falling factorial formula for $\langle\oo{x},\oo{x_0}\rangle^{\underline{p-1}}$ is divisible by $p$. That is a contradiction, since all the terms in the formula above are now divisible by $p$ except for the first one.

\smallskip
We would now like to know $\dim(\mathrm{span}\{f_{\oo{x}}\}_{\oo{x}\in I})$. A more explicit representation of $f_{\oo{x}}$
$$f_{\oo{x}}(t_1,\ldots,t_d)=\Big(\sum x_it_i\Big)\Big(\sum x_it_i-1\Big)\cdots\Big(\sum x_it_i-(p-2)\Big)$$
reveals, after opening the brackets, that $f_{\oo{x}}$ is a linear combination of monomials of degree at most $p-1$ in the $d$ variables $t_1,\ldots,t_d$. Since $t_i\in\{0,1\}$, we have $t_i^2=t_i$, so $f_{\oo{x}}$ is in fact equal to a linear combination of square-free monomials of degree at most $p-1$ in $d$ variables. The dimension of the vector space of such functions is ${d\choose 0}+\cdots+{d\choose p-1}$, where ${d\choose i}$ is the number of square-free monomials of degree $i$ (that is, products of $i$ out of $d$ variables).

\smallskip
To conclude, $\{f_{\oo{x}}\}_{\oo{x}\in I}$ is a set of linearly independent vectors in a vector space of dimension ${d\choose 0}+\cdots+{d\choose p-1}$, which means that $|I|\leq {d\choose 0}+\cdots+{d\choose p-1}$, as we wanted to prove.
\end{proof}

\begin{remark}The book \textit{Thirty-three Miniatures: Mathematical and Algorithmic Applications of Linear Algebra} by Ji\v{r}\`{i} Matou\v{s}ek is a recommended source if you are interested in algebraic tools in combinatorics (preliminary version from the author's homepage \href{http://kam.mff.cuni.cz/~matousek/stml-53-matousek-1.pdf}{\texttt{http://kam.mff.cuni.cz/\~{}matousek/stml-53-matousek-1.pdf}}). The proof above followed loosely Chapter 17.
\end{remark}

\end{document}




