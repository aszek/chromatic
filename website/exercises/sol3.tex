%
\documentclass[a4paper]{article}


\usepackage{color}              %Farben, f.r \definecolor{}
\usepackage{amssymb}            %Mathematische Symbole
\usepackage{amsthm}             %Besseres \newtheorem
\usepackage{amsmath}           %Mathematische Umgebungen
\usepackage{mathtools}          %\xRightarrow, etc
\usepackage{mathrsfs}           %enthaelt \mathscr
\usepackage[matrix,arrow,curve]{xy}     %Diagramme
\usepackage{graphicx}
\usepackage{enumerate}          % in-place numerations def.
\usepackage{fullpage}

\newcommand{\RR}{\mathbb{R}}
\newcommand{\ZZ}{\mathbb{Z}}

%% Dokument Beginn %%%%%%%%%%%%%%%%%%%%%%%%%%%%%%%%%%%%%%%%%%%%%%%%%%%%%%%%
\begin{document}
\pagestyle{empty}
\begin{center}
	{\Large\bf Graph coloring}\\
	{\large\bf Problem sheet 3 --- solutions}\\
	\line(1,0){330}
\end{center}


\begin{enumerate}
\item \texttt{FunnyGraph(99).degree\_sequence()} is \texttt{[298,292,292,...]}, which means that $\Delta=298$ and the graph has only one vertex of maximum degree. By an extension of Vizing's theorem (Lecture notes 11, Proposition 3) $\chi'=\Delta=298$. 

\item Under the usual identification with the vertices of the standard cube $[0,1]^d\subset \RR^d$ all vertices of $Q_d(u,s)$ lie in the affine hyperplane $\sum x_i =s$, which is (isometric to) $\RR^{d-1}$. 

Sage can compute $\alpha(Q_{10}(4,5))=12$, so $\chi(Q_{10}(4,5))\geq {10 \choose 5}/12 = 21$. 

\smallskip
Comments: As far as I know $\chi(\RR^9)\geq 21$ is the best known lower bound. This proof was found in an undergraduate (!) research project \texttt{http://arxiv.org/abs/1409.1278}, see Section 3. 

\item $2^d$. The vertices of the cube $[0,1]^d$ form a clique of size $2^d$ in the $\ell_\infty$-unit-distance graph, so $2^d$ colors are necessary. A coloring $c:\RR^d\to\{0,1\}^d$ can be given by
$$c((x_1,\ldots,x_d))=(\lfloor x_1\rfloor \ \textrm{mod}\ 2,\ldots, \lfloor x_d\rfloor \ \textrm{mod}\ 2 ).$$
This is a coloring since $|x_i-y_i|=1$ implies that $\lfloor x_i\rfloor, \lfloor y_i\rfloor$ are two consecutive integers, hence of opposite parity. 

\smallskip
Comments: This is just a compact way of encoding the tiling $\RR^d$ by unit cubes (it easily looks after the colors on the boundaries of the cubes). Unit distance graphs for $\ell_p$ metrics (other than Euclidean $\ell_2$) are almost completely unexplored.


\item We want to compute $c_n=P(L(P_{2\times n}), 3)$, where $L$ denotes the line graph and $P(\cdot,\cdot)$ is the usual chromatic polynomial. We can draw $L(P_{2\times n})$  and apply the deletion-contraction rule to one of the extremal (say ``leftmost'') edges of that graph. That will lead to a recursion $c_n=2c_{n-1}-6$ for $n\geq 3$. The main observation is that one of the graphs in the deletion-contraction formula has only $6$ colorings with $3$ colors (choosing the colors of two extremal vertices determines uniquely a $3$-coloring of the rest of that graph). This way we avoid computing the full chromatic polynomial.

The recursion can also be guessed by looking at the sequence\\ \texttt{graphs.Grid2dGraph(2,n).line\_graph().chromatic\_polynomial()(3)} \\
for small $n$.

From $c_n=2c_{n-1}-6$ and some initial conditions one proves easily by induction $c_n=3\cdot 2^{n}+6$ for $n\geq 2$. The case $n=1$ is special and $c_1=3$.

\smallskip
Comments: Most of you did a similar analysis directly, without looking at the intermediate line graph.


\item Thank you for active participation in the course. I hope it gave some idea about the diversity of combinatorial techniques out there.


\end{enumerate}

\end{document}




